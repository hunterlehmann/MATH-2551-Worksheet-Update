\fancyhead[C]{Section 14.6}
\fancyhead[R]{\daytwelve}

\section*{\centering Chapter 14.6: Linearization and Tangent Planes}

\textbf{D2: Tangent Planes and Linear Approximations.} I can find equations for tangent planes to surfaces and linear approximations of functions at a given point and apply these to solve problems.

\subsection*{Mechanics}
\begin{enumerate}	
    \item %14.6 # 30b
    \question{Find the linearization of $f(x,y)=e^{2y-x}$ at $(1,2)$. Without doing any more calculations, find an equation of the tangent plane of the surface $f(x,y)$ at $(1,2)$.}
    {% answer goes here
    $L(x,y)=e^3-e^3(x-1)+2e^3(y-2)$
    
    tangent plane: $z=e^3-e^3(x-1)+2e^3(y-2)$
    }
    {% solution goes here
    }
     
	
	\item \question{Find the linearization of $f(x,y,z)=\arctan(xyz)$ at $(1,1,0)$.}
    {% answer goes here
    $L(x,y,z)=z$
    }
    {% solution goes here
    } 
	\item \question{Use the linearization to approximate $f(2.95, 7.1)$ for the function $f(x,y)=\sqrt{x^2+y}$, knowing that $f(3,7)=4$.	}
    {% answer goes here
    $f(2.95, 7.1)\approx 4-1/40$
    }
    {% solution goes here
    }
    \item \question{Find an equation of the tangent plane to the unit sphere $x^2+y^2+z^2=1$ at the point $\left(\dfrac{1}{\sqrt{2}},\dfrac{1}{2},\dfrac{1}{2}\right)$.}
    {% answer goes here
        $\sqrt{2}\left( x-\dfrac{1}{\sqrt{2}}\right)+\left(y-\dfrac{1}{2}\right)+\left(z-\dfrac{1}{2}\right)=0$
    }
    {% solution goes here
    }
\end{enumerate} 
\subsection*{Applications}
\begin{enumerate}[resume]
    \item \question{Suppose you are shining a flashlight on a smooth surface. The \emph{angle of incidence}, $\theta_i$ is the angle at which a light beam hits the surface, measured with respect to the surface normal (i.e., the normal vector to the tangent plane at point of contact). 
    The \emph{angle of reflection} $\theta_r$ is the angle of the reflected light beam measured with respect to the surface normal. 
    The \emph{law of reflection} states that in a vacuum, we must have $\theta_i=\theta_r$. 
    Draw a labeled picture to convince yourself that this is reasonable.
    
    Now, consider the paraboloid $z = x^2+y^2$, and a light ray traveling along the path $\mathbf{r}(t)= (-2,-3,2)t+(3,4,0)$. 
    Compute the angle of reflection at the point $(1,1,2)$ 
    [\emph{Hint: How can one find the angle between two vectors?}]. }
    {% answer goes here
     $\theta_r = \arccos(\frac{4}{\sqrt{17}})\approx 0.245\ \text{rad}$
    }
    {% solution goes here
    $\mathbf{n}=(2,2,-1),\theta_i=\arccos(\frac{4}{\sqrt{17}})\approx 0.245\ \text{rad}$
    } 
\end{enumerate}
\subsection*{Extensions}
\begin{enumerate}[resume]	
    \item \question{Use software to graph the function $z=x^{1/3}y^{1/3}$.  
    Examine the graph at $(0,0)$ - does it look like the function has a tangent plane there? 
    Use this to deduce a necessary condition for a function $f(x,y)$ to have a tangent plane. }
    {% answer goes here
    No, there are two different tangent planes.  The function cannot have a cusp at the point with the tangent plane.
    }
    {% solution goes here
    } 
\end{enumerate}