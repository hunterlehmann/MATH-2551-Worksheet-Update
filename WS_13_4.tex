
\fancyhead[C]{Section 13.4}
\fancyhead[R]{\daysix}
\iftoggle{questions}
{\begin{center}{\large \section*{\centering Chapter 13.4: Curvature and Normals}}
\end{center}

\subsection*{Mechanics}
\begin{enumerate}	
	\item Compute the unit tangent vector, unit normal vector, and curvature of the curve $\br(t)=\langle \sqrt{2} t, 1+t, e^t\rangle$ for all $t\in\R$.
	
	\item Find $\bT, \bN$ and $\kappa$ for the curve $\br(t)=\cos(2t)\bi-\sin(2t)\bj+6t\bk$, for $t \geq  0$. What do you notice about $\kappa$? Explain (perhaps with a picture) why this happens. 
    \item Compute $\bN$ for the curve $\br(t)=\langle t, (1/3)t^3\rangle, t\in\R$ for $t\neq 0$.\\
    
	Does $\bN$ exist at $t=0$? Graph the curve, along with its normal vectors at the times $t=-1,-0.5,0.5,1$ and explain what is happening to $\bN$ as $t$ passes through $(0,0)$
\end{enumerate}
\subsection*{Applications}
\begin{enumerate}[resume]
	\item You are an engineer overseeing the construction of a certain bridge on campus. The blueprint shows that the bridge has a side view profile which looks like the parabola $y=x^2$. Unfortunately, the material that the bridge is supposed to be built with is extremely rigid, and can only support curves with $\kappa \leq 1.5$ units. Can this bridge be safely built with this material? [\textit{Hint: Where is the curvature the greatest?}]
    \item Imagine that you are an ant travelling along the space curve \begin{equation*}
        \br_1(t) = (\frac{3}{2}t^2+2t,4t-1,-3t^2+10t)
    \end{equation*} 
    while your ant-friend is travelling along a different space curve \begin{equation*}
        \br_2(t) = (2t^2-3t+10,-\frac{1}{2}t^2+9t,-2t^2)
    \end{equation*}
    Assuming you are both looking "forwards" and are on the same scale of time, is there a time $t$ when you are both looking in the same direction? If so, at what time? 
\end{enumerate}
\subsection*{Extensions}
\begin{enumerate}[resume]
    \item For a smooth curve $\br(t)$, define its \textit{binormal vector} $\mathbf{B}(t)$ at a time $t$ to be $\mathbf{B}(t)=\bT(t)\times \bN(t)$, where the $\times$ is the vector cross product. Compute $\mathbf{B}$ for $\br(t) = (t,3\cos t, 3\sin t)$. 
	\item Give an example of a parametric curve in $\mathbb{R}^2$ which has $ \bN (t) = \left(\frac{-3}{\sqrt{e^{2t}+9}},\frac{e^t}{\sqrt{e^{2t}+9}}\right)$. You may want to use the fact that $\|(e^t,3)\| =\sqrt{e^{2t}+9}$. [\textit{Hint: First deduce a possible $\bT$, then use the given fact, and integrate.}]
	
\end{enumerate}
}{}

\iftoggle{answers}{\begin{center}{\large \textbf{Math 2551 Worksheet Answers: Curvature and Normals}}
\end{center}

\begin{enumerate}	

\item $\bT(t)=\dfrac{1}{1+e^{2t}}\langle \sqrt{2}e^t,e^{2t}, -1\rangle$

$\bN(t)=\dfrac{1}{1+e^{2t}}\langle 1-e^{2t}, \sqrt{2}e^t, \sqrt{2}e^t\rangle$

$\kappa(t)=\dfrac{\sqrt{2}e^{2t}}{(e^{2t}+1)^2}$

\item $\bT(t)=\cos(t)\bi+\sin(t)\bj$

$\bN(t)=-\sin(t)\bi+\cos(t)\bj$

$\kappa(t)=\dfrac{1}{t}$

\item $\bT=\langle \dfrac{1}{\sqrt{1+t^4}}, \dfrac{t^2}{\sqrt{1+t^4}}\rangle$

$\bN=\langle \dfrac{-t^2}{\sqrt{1+t^4}}, \dfrac{1}{\sqrt{1+t^4}}\rangle$ if $t>0$ and $\langle \dfrac{t^2}{\sqrt{1+t^4}}, \dfrac{-1}{\sqrt{1+t^4}}\rangle$ if $t<0$.

The normal vector does not exist when $t=0$; as $t$ passes from negative to positive values the normal vector changes which side of the curve it is on.

\item The point of greatest curvature occurs at $x=0$. Using the 
parameterization $\br(t)=\langle t, t^2\rangle$ gives 
$\kappa(t)=\dfrac{2}{(1+4t^2)^{3/2}}$, which is maximized when $t=0$.
\end{enumerate}
}{}
\iftoggle{solutions}
{
Solutions go here in the same format.
}{}
