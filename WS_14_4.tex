
\fancyhead[C]{Section 14.4}
\fancyhead[R]{\dayten}

\section*{\centering Chapter 14.4: The Chain Rule}
\subsection*{Mechanics}
\begin{enumerate}
	\item Use the chain rule to compute the total derivatives of the following at the prescribed points. [\textit{Recall that $Df$ for $f:\mathbb{R}^n\to\mathbb{R}^m$ is an $m\times n$ matrix.}]
    \begin{enumerate}
		\item $f:\mathbb{R}\to\mathbb{R}$ given by $f(t)=h(g(t))$, where $g(t)= (t+1,t^2)$, $h(x,y)=xy$ at $t=2$.
		\item $f:\mathbb{R}^2\to\mathbb{R}$ given by $f(x,y)=(b\circ a)(x,y)$, where $a(x,y)= x\sin y$, $b(t)=8t-t^2$ at $(x,y)=(4,\pi/3)$. 
        \item $f:\mathbb{R}\to\mathbb{R}^2$ given by $f(t)=u(v(t))$, where $v(t)= (1,t,t^2)$, $u(x,y,z) = (xy,yz)$ at $t=1$. 
        \item $f:\mathbb{R}^2\to\mathbb{R}^2$ given by $f(x,y)=h(g(x,y))$, where $g(x,y)=(3x+4y,5x+7y)$, $h(u,v) = (7u-4v,-5u+3v)$ at $(x,y)=(0,0)$. 
	\end{enumerate}
	\item Find the values of $t$ where $\dfrac{dz}{dt}=0$ if $z=3x+4y$, $x=t^2$, and $y=2t$.
	
	\item Let $w(x,y,z)= xy + yz + zx$, where $x= r \cos \theta, \ \ y= r \sin \theta, \ \  z= r \theta.$ 	Find $\dfrac{\partial w}{\partial r}$ and $\dfrac{\partial w}{\partial \theta}$ when $r=2$ and $\theta = \dfrac{\pi}{2}$.

\end{enumerate}
\subsection*{Applications}
\begin{enumerate}[resume]
      \item You are a myrmecologist (ant scientist) studying the behaviour of ants on anthills. You equipped an ant with a tracker to track its motion. Unfortunately, the exact geometry of the hill, as well as the ant's precise motion are too delicate to measure. Fortunately, at time $t=5$ seconds it is known that
	\[ \pdev{z}{x}=5, \quad \pdev{z}{y}=-2, \quad \dfrac{dx}{dt}=3, \quad \dfrac{dy}{dt}=7. \] where $z$ denotes the height of the ant relative to ground height.
	Use this information to determine if the ant is going uphill or downhill (or neither) at $t=5$ seconds. 
    \item The multivariable chain rule is a key tool in modern machine learning. In the big picture, neural networks "learn" parameters through an algorithm called \textit{gradient descent}. This algorithm involves computing total derivatives of long chains of functions in high dimensions, which is in general extremely hard to do. The chain rule tells us that instead of contending with such a long chain of functions all at once, one can instead study each "layer" by itself, then combine everything with matrix multiplication, which is relatively easier. This is called \textit{backpropagation}. 
	
\end{enumerate}
\subsection*{Extensions}
\begin{enumerate}[resume]
    \item Suppose we have a differentiable function $w=g(x,y)$ and $x$ and $y$ are differentiable functions of $t$ and we know the following information.
	\[g(1,0)=1,\ g_x(1,0)=-2,\ g_y(1,0)=2,\ g(-1,2)=3,\ g_x(-1,2)=1,\ g_y(-1,2)=-2,\]
	\[ x(2)=1,\ y(2)=0,\ x(1)=1,\ y(1)=3,\ x'(2)=4,\ y'(2)=-1,\ x'(1)=0,\ y'(1)=2 \]
	
	If possible, find $\dfrac{dw}{dt}(1)$ and $\dfrac{dw}{dt}(2)$ or explain why the given information is not enough to do so.  Which of these pieces of information would you not use at all to compute either value?
    \item Give an example of a nonconstant, differentiable function $f:\mathbb{R}^2\to\mathbb{R}^2$ for which $Df$ is \textbf{not} invertible at $(0,0)$, but invertible at $(1,0)$.
\end{enumerate}
{}

\iftoggle{answers}{
\begin{center}{\large \textbf{Math 2551 Worksheet Answers: Chain Rule}}
\end{center}

\begin{enumerate}
	\item $\dfrac{dz}{dt}(t_0)=1$
	
	\item $t=-4/3$
	
	\item $\dfrac{\partial w}{\partial r}(2,\pi/2)=2\pi$ \qquad $\dfrac{\partial w}{\partial \theta}(2,\pi/2)=-2\pi$
	
	\item $\dfrac{dw}{dt}(2)=-10$.  $\dfrac{dw}{dt}(1)$ cannot be computed from the given information because we do not know the values of $g_x$ or $g_y$ at $(x(1),y(1))=(1,3)$.  We do not use the values of $g(1,0),g(-1,2),g_x(-1,2),g_y(-1,2)$.
\end{enumerate}

}{}
\iftoggle{solutions}
{
Solutions go here in the same format.
}{}
