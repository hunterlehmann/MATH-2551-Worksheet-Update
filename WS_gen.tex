\DocumentMetadata{
	lang        = en-US,
	pdfstandard = ua-2,
	pdfstandard = a-4f, %or a-4
	tagging=on,
	tagging-setup={math/setup={mathml-SE,mathml-AF}}%, math/alt/use=true} 
}
\documentclass[12pt]{article}
\usepackage{unicode-math}
\usepackage{amsmath}
\usepackage{theorem}
\usepackage{multicol}
\usepackage{graphicx}
\usepackage{etoolbox}
\usepackage{pgfplots}
%\usepackage{todonotes}
%\usepackage{tcolorbox}
%\usepackage{enumitem}
%\usepackage{verbatimbox}

\pgfplotsset{compat=1.18}
\topmargin-1cm \textheight 23cm \textwidth  17cm
\oddsidemargin-.1cm
\evensidemargin-.1cm
\parindent 0mm
\marginparwidth 2cm
\parskip 1.3ex plus 0.5ex minus 0.5ex
%%%%%%%%%%%%%%%%%%%%%%%%%%%%%%%%%%%%%%%%%%%%%%%
\usepackage{fancyhdr}
\usepackage{harpoon}
\usepackage{graphicx}
\usepackage{hyperref}
\usepackage{tikz}
\pagestyle{fancy}
%... then configure it.
\setlength{\headheight}{14.49998pt}
\fancyhead{} % clear all header fields
\fancyhead[L]{Math 2551}
\fancyfoot{} % clear all footer fields
\fancyfoot[C]{\thepage}

% update every semester
\newcommand{\semester}{Fall 2024}

\newcommand{\dayone}{Day One}
\newcommand{\daytwo}{Day Two}
\newcommand{\daythree}{Day Three}
\newcommand{\dayfour}{Day Four}
\newcommand{\dayfive}{Day Five}
\newcommand{\daysix}{Day Six}
\newcommand{\dayseven}{Day Seven}
\newcommand{\dayeight}{Day Eight}
\newcommand{\daynine}{Day Nine}
\newcommand{\dayten}{Day Ten}
\newcommand{\dayeleven}{Day Eleven}
\newcommand{\daytwelve}{Day Twelve}
\newcommand{\daythirteen}{Day Thirteen}
\newcommand{\dayfourteen}{Day Fourteen}
\newcommand{\dayfifteen}{Day Fifteen}
\newcommand{\daysixteen}{Day Sixteen}
\newcommand{\dayseventeen}{Day Seventeen}
\newcommand{\dayeighteen}{Day Eighteen}
\newcommand{\daynineteen}{Day Nineteen}
\newcommand{\daytwenty}{Day Twenty}
\newcommand{\daytwentyone}{Day Twenty-One}
\newcommand{\daytwentytwo}{Day Twenty-Two}
\newcommand{\daytwentythree}{Day Twenty-Three}
\newcommand{\daytwentyfour}{Day Twenty-Four}
\newcommand{\daytwentyfive}{Day Twenty-Five}
\newcommand{\daytwentysix}{Day Twenty-Six}
\newcommand{\daytwentyseven}{Day Twenty-Seven}
\newcommand{\daytwentyeight}{Day Twenty-Eight}

\newcommand{\vect}[1]{\overrightharp{#1}}
\DeclareMathOperator{\proj}{proj}
\newcommand{\proje}[2]{\proj_{\mathbf{#1}} \mathbf{#2}}
\newcommand{\dotp}{\boldsymbol{\cdot}}
\newcommand{\vecf}[3]{#1 \bi #2 \bj #3 \bk}	
\newcommand{\pdev}[2]{\dfrac{\partial #1}{\partial #2}}	
\newcommand{\bi}{\mathbf{i}}
\newcommand{\bj}{\mathbf{j}}
\newcommand{\bk}{\mathbf{k}}
\newcommand{\br}{\mathbf{r}}
\newcommand{\bv}{\mathbf{v}}
\newcommand{\ba}{\mathbf{a}}
\newcommand{\bff}{\mathbf{f}}
\newcommand{\bg}{\mathbf{g}}
\newcommand{\bh}{\mathbf{h}}
\newcommand{\bn}{\mathbf{n}}
\newcommand{\bu}{\mathbf{u}}
\newcommand{\bN}{\mathbf{N}}
\newcommand{\bT}{\mathbf{T}}
\newcommand{\bF}{\mathbf{F}}
\DeclareMathOperator{\curl}{curl}
\DeclareMathOperator{\Div}{div}

\newcommand{\R}{\mathbb{R}}
\newcommand{\N}{\mathbb{N}}
\newcommand{\Z}{\mathbb{Z}}
\newcommand{\Q}{\mathbb{Q}}
\newcommand{\Tx}{\textstyle}
\newcommand{\Ds}{\displaystyle}

\newtoggle{questions}
\newtoggle{solutions}
\newtoggle{answers}
\toggletrue{questions}   %print questions
%\toggletrue{answers}  %print answers
%\toggletrue{solutions} %print solutions

\newcommand{\question}[3]{\iftoggle{questions}{#1}{}

\iftoggle{answers}{\textbf{Answer:} #2}{}

\iftoggle{solutions}{\textbf{Solution:} #3}{}}

\begin{document}

% \begin{myverbbox}{\verbSection}
% \section{\centering stuff}
% \end{myverbbox}
% \begin{myverbbox}{\verbCurrent}
% \begin{center}
%      {\large \textbf{stuff}}
% \end{center}
% \end{myverbbox}
% \begin{myverbbox}{\verbQuestion}
% \question{question goes here}
%     {final answer goes here}
%     {solution goes here}
% \end{myverbbox}

% Things to work on:

% \todo[inline]{Add solutions to all worksheet days}
% \todo[inline]{Update problems to better ones as needed}

% \todo[inline]{Change all headers to be 

% \begin{minipage}{0.2\textwidth}
% \verbSection
% \end{minipage}

% instead of 

% \begin{minipage}{0.2\textwidth}
% \verbCurrent
% \end{minipage}
% }

% \todo[inline]{Add Learning Outcome information to the top of each worksheet as appropriate (starting w/ 12.5).  The Learning Outcomes are listed below the end document here.  Each worksheet should get at the top ``The Learning Outcomes associated with this worksheet are:'' followed by a list of all of the appropriate ones (usually just one or two).}

% \todo[inline]{Use new \begin{minipage}{0.2\textwidth}
%     \verbQuestion
% \end{minipage}

% format instead of the large blocks}

%	
\fancyhead[C]{Section 12.1}
\fancyhead[R]{\daytwo}

\iftoggle{questions}{
\begin{center}{\large \section*{\centering Chapter 12.1: The Geometry of $\R^3$}}
\end{center}
\newcounter{count}
%\section{\centering {Math 2551 Worksheet: $\R^3$}}
\section{Mechanics}
\begin{tcolorbox}
    \begin{enumerate}
    
	\item Find the components of the vector	in 3-space of length 3 lying in the $yz$-plane pointing upward at an angle of $\pi/6$ measured from the positive $y$-axis.

	\item Which is traveling faster, a car whose velocity vector is $\langle 28,33 \rangle$ km/h or a car whose velocity vector is $\langle 40, 0 \rangle$ km/h?  At what speed is the faster car traveling?
	
	\item Find the equation of a sphere of radius $4$ in $\R^3$ centered at the point $(1,-2,3)$.
	\setcounter{count}{\value{enum}}
\end{enumerate}
\end{tcolorbox}

\section{Applications}
\begin{tcolorbox}
    \begin{enumerate}
    \setcounter{enum}{\value{count}}
	\item Find the components of the vector	in 3-space of length 3 lying in the $yz$-plane pointing upward at an angle of $\pi/6$ measured from the positive $y$-axis.

	\item Which is traveling faster, a car whose velocity vector is $\langle 28,33 \rangle$ km/h or a car whose velocity vector is $\langle 40, 0 \rangle$ km/h?  At what speed is the faster car traveling?
	
	\item Find the equation of a sphere of radius $4$ in $\R^3$ centered at the point $(1,-2,3)$.
	
\end{enumerate}
\end{tcolorbox}
\section{Extensions}

}{}

\iftoggle{answers}{\begin{center}{\large \textbf{Math 2551 Worksheet Answers: $\R^3$}}
\end{center}


\begin{enumerate}
	
	\item $x=0, y= 3\sqrt{3}/2, z=3/2$.
	
	\item The first car is faster, at approx 43.28 km/h.
	
	\item $(x-1)^2+(y+2)^2+(z-3)^2=16$.
	
\end{enumerate}
}{}
\iftoggle{solutions}
{
Solutions go here in the same format.
}{}

% 	
\fancyhead[C]{Section 12.1}
\fancyhead[R]{\daytwo}

\iftoggle{questions}{
\begin{center}{\large \section*{\centering Chapter 12.1: The Geometry of $\R^3$}}
\end{center}
\newcounter{count}
%\section{\centering {Math 2551 Worksheet: $\R^3$}}
\section{Mechanics}
\begin{tcolorbox}
    \begin{enumerate}
    
	\item Find the components of the vector	in 3-space of length 3 lying in the $yz$-plane pointing upward at an angle of $\pi/6$ measured from the positive $y$-axis.

	\item Which is traveling faster, a car whose velocity vector is $\langle 28,33 \rangle$ km/h or a car whose velocity vector is $\langle 40, 0 \rangle$ km/h?  At what speed is the faster car traveling?
	
	\item Find the equation of a sphere of radius $4$ in $\R^3$ centered at the point $(1,-2,3)$.
	\setcounter{count}{\value{enum}}
\end{enumerate}
\end{tcolorbox}

\section{Applications}
\begin{tcolorbox}
    \begin{enumerate}
    \setcounter{enum}{\value{count}}
	\item Find the components of the vector	in 3-space of length 3 lying in the $yz$-plane pointing upward at an angle of $\pi/6$ measured from the positive $y$-axis.

	\item Which is traveling faster, a car whose velocity vector is $\langle 28,33 \rangle$ km/h or a car whose velocity vector is $\langle 40, 0 \rangle$ km/h?  At what speed is the faster car traveling?
	
	\item Find the equation of a sphere of radius $4$ in $\R^3$ centered at the point $(1,-2,3)$.
	
\end{enumerate}
\end{tcolorbox}
\section{Extensions}

}{}

\iftoggle{answers}{\begin{center}{\large \textbf{Math 2551 Worksheet Answers: $\R^3$}}
\end{center}


\begin{enumerate}
	
	\item $x=0, y= 3\sqrt{3}/2, z=3/2$.
	
	\item The first car is faster, at approx 43.28 km/h.
	
	\item $(x-1)^2+(y+2)^2+(z-3)^2=16$.
	
\end{enumerate}
}{}
\iftoggle{solutions}
{
Solutions go here in the same format.
}{}

% 	
\fancyhead[C]{Section 12.4}
\fancyhead[R]{\daytwo}

\iftoggle{questions}{
\begin{center}{\large \textbf{Math 2551 Worksheet: Cross Products}}
\end{center}


\begin{enumerate}
	
	
	\item Let $P=(1,-1,2)$, $Q=(2,0,-1)$, and $R=(0,2,1)$. 
	\begin{enumerate}
		\item Find the area of the triangle determined by the points $P,Q$, and $R$.
		
		\item Find a unit vector normal to the plane containing $P$, $Q$, and $R$.
	\end{enumerate}
	
	\item If the statement is \textit{always} true, answer true.  If the statement is \textit{ever} false, answer false.  Justify your answer.
	
	In each case, $\bf{u}$, ${\bf v}$, and $\bf{w}$ are vectors in $\mathbb{R}^3$.
	\begin{enumerate}
		\item $\bf{u} \cdot \bf{v} = \bf{v} \cdot \bf{u}$
		\item $\bf{u}\cdot \bf{u} = |\bf{u}|^2$
		\item $(\bf{u} \times \bf{u}) \cdot \bf{u}=0$
		
	\end{enumerate} 
	
	\item Suppose ${\bf u}$, ${\bf v}$, and ${\bf w}$ are vectors in $\R^3$.  Which of the following make sense, and which do not?
	For those that make sense, is the result a vector or a scalar?
	\begin{enumerate}
		\item $({\bf u} \times {\bf v}) \cdot {\bf w}$ 
		\item ${\bf u} \times ({\bf v} \cdot {\bf w})$
		\item ${\bf u} \times ({\bf v} \times {\bf w})$ 
		\item ${\bf u} \cdot ({\bf v} \cdot {\bf w})$
	\end{enumerate}

\item Let ${\bf u} = \langle 2, 3 \rangle$.  Find the maximum possible value for ${\bf u} \cdot {\bf v}$ if ${\bf v}$ is a unit vector,
and find a ${\bf v}$ which gives this maximum.  Then repeat the problem with ``maximum" replaced by ``minimum."
\end{enumerate}
}{}

\iftoggle{answers}{\begin{center}{\large \textbf{Math 2551 Worksheet 2 Answers: $\R^3$ and Cross Products}}
\end{center}


\begin{enumerate}
	
	
	\item Let $P=(1,-1,2)$, $Q=(2,0,-1)$, and $R=(0,2,1)$. 
	\begin{enumerate}
		\item $2\sqrt{6}$
		
		\item $\dfrac{1}{\sqrt{6}}\langle 2, 1, 1 \rangle$
	\end{enumerate}
	
	\item \begin{enumerate}
		\item True
		\item True 
		\item True
		
	\end{enumerate} 
	
	\item \begin{enumerate}
		\item Makes sense, scalar
		\item Does not make sense
		\item Makes sense, vector
		\item Does not make sense
	\end{enumerate}
	
	\item Max value is $\sqrt{13}$, given by $\bv=\bu/|\bu|$ and min value is $-\sqrt{13}$, given by $\bv=-\bu/|\bu|$.
\end{enumerate}
}{}
\iftoggle{solutions}
{
Solutions go here in the same format.
}{}

% 
\fancyhead[C]{Section 12.5}
\fancyhead[R]{\daythree}

\iftoggle{questions}{
\begin{center}{\large \textbf{Math 2551 Worksheet: Lines and Planes}}
\end{center}


\begin{enumerate}
	
	\item Find an equation for 
	\begin{enumerate}
		\item the line through point $P=(1,2,-1)$ and point $Q=(-1,0,1)$.
		
		\item the line through $(0,-7,0)$ perpendicular to the plane $x+2y+2z=13$.
		
		\item the line in which the planes $3x-6y-2z=3$ and $2x+y-2z=2$ intersect.
	\end{enumerate}
	
	\item Find a vector in the direction of the line of intersection $\ell$ of the planes $2x+y-z=3$ and  $x+2y+z=2.$  Find a plane which goes	through $(2,1,-1)$ and is perpendicular to $\ell$ (and thus both planes).  
	
	\item Find 2 planes that are not parallel that both contain the points $P(1,-1,1)$, $Q(3,2,0)$, and $R(5,5,-1)$. When will 3 distinct points NOT determine a unique plane? 
	
	\item Find the point where the line $\br(t)=\langle 2, 3+2t, 1+t\rangle$  intersects the plane
	$2x-y+3z=6$.
	
	\item  %12.5 68
	How can you tell when two planes $A_1x + B_1 y + C_1 z = D_1$ and  $A_2x + B_2 y + C_2 z = D_2$ are parallel?  Perpendicular?
	Justify your answer. 
	
	\item Find the point at which the lines $\ell_1(t)=\langle 
	2,3,1\rangle+\langle 1,-1,1\rangle t$ and $\ell_2(t)=\langle 2,1,-2\rangle 
	t+\langle 6,2,1\rangle$ intersect.
	
%	\item Recall from linear algebra that the \textbf{projection} of a vector 
%$\bu$ onto $\bv$ is the component of $\bu$ in the direction of $\bv: 
%\proj_{\bv}\bu=\dfrac{\bu\cdot\bv}{|\bv|^2}\bv$.
%	
%	\begin{enumerate}
%		\item The distance from a point $P$ to a plane is the shortest distance 
%from $P$ to any point on the plane.  Use this and the above to compute the 
%distance from the point $P=(1,2,3)$ to the plane $2x-y+3z=5$.\\
%		
%		\includegraphics[scale=0.6]{12_5_plane_dist.png}
%		
%		\item The distance from a point $P$ to a line is the shortest distance 
%from $P$ to any point on the line.  Use this and a well-chosen cross product 
%to 
%compute the distance from the point $P=(-1,2,1)$ to the line $\langle 
%1,1,1\rangle+t\langle 2,3,-1\rangle$.\\
%		
%		\includegraphics[scale=0.7]{12_5_line_dist.png}
%	\end{enumerate}

\end{enumerate}
}{}

\iftoggle{answers}{
\begin{center}{\large \textbf{Math 2551 Worksheet Answers: Lines and Planes}}
\end{center}


\begin{enumerate}
	
	\item 
	\begin{enumerate}
		\item $\br(t)=\langle 1 -2t, 2-2t, -1+2t\rangle$
		
		\item $\br(t)=\langle t, -7+2t,2t\rangle$
		
		\item $\br(t)=\langle 1+14t, 2t, 15t\rangle$
	\end{enumerate}
	
	\item vector: $3\bi-3\bj+3\bk$.
	
	plane: $3(x-2)-3(y-1)+3(z+1)=0$
	
	\item many correct solutions; pick two distinct points off of the line $PQ$ which are not collinear with the given points.  Two such planes are $-2(x-1)+3(y+1)+5(z-1)=0$ and $-3(x-1)+3(y+1)+3(z-1)=0$.
	
	\item $(2,7,3)$
	
	\item  Parallel: $\langle A_1, B_1, C_1\rangle = \lambda \langle A_2, B_2, C_2\rangle$ for some $\lambda\neq 0$
	
	Perpendicular:  $\langle A_1, B_1, C_1\rangle \dotp \langle A_2, B_2, C_2\rangle=0$
	
	\item $(4,1,3)$
%	\item \begin{enumerate}
%		\item distance $= \dfrac{\vec{QP}\cdot\bn}{\bn}$ with $Q=(1,0,1)$ (any 
%point on the plane works).  So the distance is $\dfrac{\langle 
%0,2,2\rangle\cdot\langle2,-1,3\rangle}{|\langle 
%2,-1,3\rangle|}=\dfrac{4}{\sqrt{14}}$
%		
%		\item distance is $|\vec{QP}|\sin(\theta)=\dfrac{|\vec{QP}\times 
%\bv|}{|\bv|}$ where $Q=(1,1,1), \bv=\langle 2,3,-1\rangle$ (any point on the 
%line and any direction vector works). So the distance is 
%$\dfrac{\sqrt{69}}{\sqrt{14}}$.
%	\end{enumerate}
\end{enumerate}
}{}
\iftoggle{solutions}
{
Solutions go here in the same format.
}{}

% 
\fancyhead[C]{Section 12.6}
\fancyhead[R]{\dayfour}

\iftoggle{questions}{
\begin{center}{\large \textbf{Math 2551 Worksheet: Quadric Surfaces}}
\end{center}


\begin{enumerate}
	
	\item For the following, identify and describe (for example, which way is it oriented? what are the cross-sections?) the type of surface.
	
	\begin{enumerate}
		\item $x^2+4z^2=16$.
		
		\item $9x^2+z^2+y^2=9$.
		
		\item $y^2=3x^2+3z^2$.
		
		\item $z=x^2+y^2+4$.
		
		\item $x^2+y^2=16-z^2$
		
	\end{enumerate}
\end{enumerate}
}{}

\iftoggle{answers}{
\fancyhead[R]{\dayfour}
\begin{center}{\large \textbf{Math 2551 Worksheet Answers: Quadric Surfaces}}
\end{center}


\begin{enumerate}
	\item 	
	\begin{enumerate}
		\item elliptical cylinder, oriented along the $y$-axis, cross-sections are ellipses in the $y=k$ planes or vertical/horizontal lines in the $x=k$ and $z=k$ planes.
		
		\item ellipsoid, centered at the origin, wider in the $z$ and $y$ directions than the $x$ direction, cross-sections are circles in the $x=k$ planes (if $k<1$), ellipses in the $z=k$ and $y=k$ planes ($k<3$)
		
		\item circular cone, oriented along the $y$-axis, cross sections are circles in the $y=k$ planes and lines in the $x=k$ and $z=k$ planes
		
		\item elliptical paraboloid, oriented in the positive $z$ direction, shifted up $4$ units, cross-sections are circles in the $z=k$ planes for $k\geq 4$ and parabolas in the $x=k$ or $y=k$ planes
		
		\item sphere, centered at $(0,0,0)$, radius $4$, cross-sections are circles in $x=k, y=k,$ and $z=k$ planes for $0\leq k\leq 4$.
		
	\end{enumerate}
\end{enumerate}
}{}
\iftoggle{solutions}
{
Solutions go here in the same format.
}{}

% 
\fancyhead[C]{Section 13.1}
\fancyhead[R]{\dayfour}

\iftoggle{questions}{
\begin{center}{\large \textbf{Math 2551 Worksheet: Curves in Space and Their Tangents}}
\end{center}


\begin{enumerate}

	\item Describe the graph of the curve $\br(t)=\langle t\cos(t), t\sin(t), t\rangle$, $t\in\R$.\\
	
	\item Find a vector-valued function for the curve of intersection of the cylinder $x^2+y^2=9$ and the plane $y+z=2$.
	
	\textit{Hint: How could you parameterize the circle $x^2+y^2=9$ in the plane?}\\
	
	\item What is the difference between the parameteric curves $\bff(t)=\langle t, t,t^2\rangle, \bg(t)=\langle t^2, t^2, t^4\rangle$, and $\bh(t)=\langle \sin(t),\sin(t),\sin^2(t)\rangle$ as $t$ runs over all real numbers?\\
	
	\item With a parametric plot and a set of $t$ values, we can associate a `direction'.  For example, the curve $\langle \cos(t),\sin(t)\rangle$, $t\in[0,2\pi]$ is the unit circle traced counterclockwise.  How can we change a set of given parametric equations and $t$ values to get the same curve, only traced backwards?
	
	\item The motion of a particle in the $xy$-plane at time $t$ is described by the vector function
	\[\br(t)= e^{t}\bi + \frac{2}{9}e^{2t}\bj\]
	\begin{enumerate}
		\item Find an equation in $x$ and $y$ whose graph is the path of the particle. Consider how $y(t)$ is related to $x(t)$ and what values $x(t)$ takes on.
		
		\item Find the particle's velocity and acceleration vectors at $t=\ln(3).$
		
		\item Sketch the path of the particle and include the particle's velocity and acceleration vectors at $t=\ln(3).$
	\end{enumerate}
	
	\item Find the parametric equations for the line that is tangent to the curve \\
	\[\vect r(t)=\left\langle \ln t, \frac{t-1}{t+2}, t \ln t \right\rangle, \text{ at } t = 1.\]
	
	\item Determine the point at which $\bff(t)=\langle t, t^2,t^3\rangle$ and $\bg(t)=\langle \cos(t), \cos(2t), t+1\rangle$ intersect, and find the angle between the curves at that point. (Hint: You'll need to set this up like the line intersection problems you've seen before, writing one in $s$ and one in $t$). 
	
	If these two functions were the trajectories of two bumblebees on the same scale of time, would the bees collide at their point of intersection? Explain.
	
	\item Find the equation of the plane perpendicular to the curve $\langle \cos(t),\sin(t),\cos(6t)\rangle$ when $t=\pi/4$.
\end{enumerate}
}{}

\iftoggle{answers}{
\fancyhead[R]{\dayfour}
\begin{center}{\large \textbf{Math 2551 Worksheet Answers: Curves in Space and Their Tangents}}
\end{center}


\begin{enumerate}

	\item This curve's graph is a spiral, narrowing to a point at the origin when $t=0$ and widening outward around the $z$-axis for larger/smaller $t$.\\
	
	\item $\br(t)=\langle 3\cos(t), 3\sin(t),2-3\sin(t)\rangle, 0\leq t\leq 2\pi$.\\
	
	\item  All three functions describe part of the same set of points in $\R^3$, which lie above the line $y=x$ in the $xy$-plane and form a parabola in the plane $x=y$.  $\bff$ traces out all of the points on this parabola, $\bg$ only those in the first octant, and $\bh$ only those which lie above the square $[-1,1]\times[-1,1]$.\\
	
	\item Many possible answers; depending on the domain and functions involved.  If the domain is bounded, e.g. $[a,b]$, then letting $s=b+(a-b)t$ and taking $\br(s)$ as the new parametric equations works. If the domain is $(-\infty,\infty)$, we can just let $s=-t$.
	
	
	\item 
	\begin{enumerate}
		\item $y=\dfrac{2}{9}x^2$ for $x>0$
		
		\item $\bv(\ln(3))=3\bi+4\bj$
		
		$\ba(\ln(3))=3\bi+8\bj$.
	\end{enumerate}
	
	\item $x(s)=s, y(s)=\dfrac{s}{3}, z(s)=s$
	
	
	
	\item $(1,1,1)$ (where the first parameter is 1 and the second is 0).  The angle is $\arccos(3/\sqrt{14})$.  The bees would not collide, since the first bee reaches the point at $t=1$ and the second bee at $t=0$.
	
	\item $-\dfrac{1}{\sqrt{2}}(x-\dfrac{1}{\sqrt{2}})+\dfrac{1}{\sqrt{2}}(y-\dfrac{1}{\sqrt{2}})+6(z-0)=0$
	
	OR $x-y-6\sqrt{2}z=0$
\end{enumerate}
}{}
\iftoggle{solutions}
{
Solutions go here in the same format.
}{}

% 
\fancyhead[C]{Section 13.2}
\fancyhead[R]{\dayfive}

\iftoggle{questions}{\begin{center}{\large \textbf{Math 2551 Worksheet: Integrals of Vector Valued Functions}}
\end{center}

\begin{enumerate}
	
	
	
	
	\item Suppose that $\br(t)$ satisfies 
	\[
	\br''(t)=-\bi-\bj-\bk, \quad t\geq 0, \qquad \br'(0)=5 \bi, \qquad \br(0)=10\bi+10\bj+10\bk
	\]
	Find $\br(t)$.
	
	\item A baseball is hit when it is $2.5$ ft above the ground. It leaves the bat with an initial velocity of $140$ ft/sec at a launch angle of $30^\circ$. At the instant the ball is hit, an instantaneous gust of wind blows against the ball, adding a component of $-14 \hat{i}$ (ft/sec) to the ball's initial velocity. A $15$ ft high fence lies $400$ ft from the home plate in the direction of the flight. (Note that gravity, g $= 32$ ft/sec$^2$)
	
	\begin{enumerate}
		\item Include an appropriate sketch.
		
		\item Find a vector equation for the path of the baseball.
		
		\item How high does the baseball go, and when does it reach maximum height?
		
		\item Find the range and flight time of the baseball, assuming that the ball is not caught.
		
		\item When is the baseball $20$ ft high? How far (ground distance) is the baseball from home plate at that height?
		
		\item Has the batter hit a home run? Explain.
	\end{enumerate}	
	
\end{enumerate}
}{}

\iftoggle{answers}{
\fancyhead[R]{\dayfive}
\begin{center}{\large \textbf{Math 2551 Worksheet 5 Answers: Calculus of Vector-Valued Functions}}
\end{center}

\begin{enumerate}
	
	\item $\br(t)=\langle -\dfrac{1}{2}t^2+5t+10,-\dfrac{1}{2}t^2+10,-\dfrac{1}{2}t^2+10\rangle, \quad t\geq 0$.
	\item A baseball is hit when it is $2.5$ ft above the ground. It leaves the bat with an initial velocity of $140$ ft/sec at a launch angle of $30^\circ$. At the instant the ball is hit, an instantaneous gust of wind blows against the ball, adding a component of $-14 \hat{i}$ (ft/sec) to the ball's initial velocity. A $15$ ft high fence lies $400$ ft from the home plate in the direction of the flight. (Note that gravity, g $= 32$ ft/sec$^2$)
	
	\begin{enumerate}
		\item Sorry, no sketch. :)
		
		\item $\vect r(t) = (140\cos 30^\circ-14)t \hat{i}+ (2.5+(140\sin 30^\circ)t-16t^2)\hat{j} = (70\sqrt{3} -  14) t \hat{i} + (2.5+70t-16t^2) \hat{j}$.
		
		\item $y_{\text{max}}=\frac{(140 \sin 30^\circ)^2}{64}+2.5=\frac{70^{2}}{64}+2.5 = 79.0625$ ft., which is reached at $t = \frac{140 \sin 30^\circ}{32}=\frac{70}{32}=2.1875$ s.
		
		\item For the time, solve $y=2.5+70t-16t^2=0$ for $t$. Using quadratic formula, we have $t=4.41$s. Then, the range at $t=4.41$ is $x(4.41) = (140\cos 30^\circ-14)(4.41)=472.94$ ft. 
		
		\item For the time, solve $y=2.5+70t-16t^2=20$ for $t$. Using quadratic formula, we have $t=0.27, \ 4.11$ seconds. Then, the range at those times are $x(0.27) = 29$ ft and $x(4.11)=441$ ft.
		
		\item Yes, according to part (d), the ball is still 20 feet above the ground when it is 441 feet from home plate.
	\end{enumerate}	

\end{enumerate}
}{}
\iftoggle{solutions}
{
Solutions go here in the same format.
}{}

% 
\fancyhead[C]{Section 13.3}
\fancyhead[R]{\dayfive}

\iftoggle{questions}{\begin{center}{\large \textbf{Math 2551 Worksheet: Arc Length}}
\end{center}

\begin{enumerate}	

	\item Let $\br(t)=\langle 6 \sin 2t, 6 \cos 2t, 5t\rangle $. Find the unit tangent vector of $\br(t)$ and find the length of the portion of the graph of $\br(t)$ where $0 \leq t \leq \pi$.
	
	
	\item Find the point on the curve
	\[
	\br(t) = (5 \sin t)\bi+(5 \cos t)\bj+12t\bk
	\]
	at a distance $26\pi$ units along the curve from the point $(0,5,0)$ in the direction of increasing arc length.
	
	\item Suppose an object's position is given by $\br(t)= (2\ln (t+1))\bi+(e^{2t}+t)\bj+(\sin^2(t))\bk$. Set up but do not evaluate the appropriate integral with limits to find the distance the object traveled from the point $A(0,1,0)$ to the point $B(\ln 4,e^2+1,\sin^2(1))$.
	
	\item Find the length of the curve 
	\[
	\br(t) = \langle\sqrt{2}t,\sqrt{3}t,(1-t)\rangle
	\]
	from $(0,0,1)$ to $(\sqrt{2}, \sqrt{3}, 0)$.
\end{enumerate}
}{}

\iftoggle{answers}{\begin{center}{\large \textbf{Math 2551 Worksheet Answers: Arc Length}}
\end{center}

\begin{enumerate}	


\item $\bT(t)=\dfrac{1}{13}\langle 12\cos(2t),-12\sin(2t),5\rangle $

length: $13\pi$


\item $(0,5,24\pi)$

\item $\Ds \int_0^1 \sqrt{(\dfrac{2}{t+1})^2+(2e^{2t}+1)^2+(2\sin(t)\cos(t))^2}\ dt$

\item $\sqrt{6}$

\end{enumerate}
}{}
\iftoggle{solutions}
{
Solutions go here in the same format.
}{}

% 
\fancyhead[C]{Section 13.4}
\fancyhead[R]{\daysix}

\iftoggle{questions}{\begin{center}{\large \textbf{Math 2551 Worksheet: Curvature and Normals}}
\end{center}

\begin{enumerate}	
	\item Find the unit tangent vector, unit normal vector, and curvature of the curve $\br(t)=\langle \sqrt{2} t, e^t, e^{-t}\rangle$, $t\in\R$.
	
	\item Find $\bT, \bN$ and $\kappa$ for the space curve $\br(t)=(\cos(t)+t\sin(t))\bi+(\sin(t)-t\cos(t))\bj+3\bk$ with $t\geq 0$.
	
	\item Compute $\bT$ and $\bN$ for the curve $\br(t)=\langle t, (1/3)t^3\rangle, t\in\R$ for $t\neq 0$.  \\
	
	Does $\bN$ exist at $t=0$? Graph the curve and explain what is happening to $\bN$ as $t$ passes from negative to positive values.
	
	\item Before doing any computations, where do you think that the curvature of the parabola $y=x^2$ is greatest?\\
	
	Compute its curvature and find the point with greatest curvature.
	
\end{enumerate}
}{}

\iftoggle{answers}{\begin{center}{\large \textbf{Math 2551 Worksheet Answers: Curvature and Normals}}
\end{center}

\begin{enumerate}	

\item $\bT(t)=\dfrac{1}{1+e^{2t}}\langle \sqrt{2}e^t,e^{2t}, -1\rangle$

$\bN(t)=\dfrac{1}{1+e^{2t}}\langle 1-e^{2t}, \sqrt{2}e^t, \sqrt{2}e^t\rangle$

$\kappa(t)=\dfrac{\sqrt{2}e^{2t}}{(e^{2t}+1)^2}$

\item $\bT(t)=\cos(t)\bi+\sin(t)\bj$

$\bN(t)=-\sin(t)\bi+\cos(t)\bj$

$\kappa(t)=\dfrac{1}{t}$

\item $\bT=\langle \dfrac{1}{\sqrt{1+t^4}}, \dfrac{t^2}{\sqrt{1+t^4}}\rangle$

$\bN=\langle \dfrac{-t^2}{\sqrt{1+t^4}}, \dfrac{1}{\sqrt{1+t^4}}\rangle$ if $t>0$ and $\langle \dfrac{t^2}{\sqrt{1+t^4}}, \dfrac{-1}{\sqrt{1+t^4}}\rangle$ if $t<0$.

The normal vector does not exist when $t=0$; as $t$ passes from negative to positive values the normal vector changes which side of the curve it is on.

\item The point of greatest curvature occurs at $x=0$. Using the 
parameterization $\br(t)=\langle t, t^2\rangle$ gives 
$\kappa(t)=\dfrac{2}{(1+4t^2)^{3/2}}$, which is maximized when $t=0$.
\end{enumerate}
}{}
\iftoggle{solutions}
{
Solutions go here in the same format.
}{}

% 
\fancyhead[C]{Section 14.1}
\fancyhead[R]{\dayseven}
\iftoggle{questions}
{\begin{center}{\large \section*{\centering Chapter 14.1: Multivariate Functions}}
\end{center}
\subsection*{Mechanics}
\begin{enumerate}	
	\item Algebraically describe the domains of each of the following functions. Then sketch them on (separate) $xy$-planes. 
	\begin{enumerate}
		\item $f(x,y)=\sqrt{x-y-1}.$
		
		\item $f(x,y) = \sqrt{(x-4)(y^2-1)}.$
		
		\item $f(x,y)=\cos^{-1}(y-4x^2)$.
		
		\item $f(x,y)= \dfrac{1}{4-x^2-y^2}.$
		
		\item $f(x,y)=\dfrac{1}{\ln(4-x^2-y^2)}$
	\end{enumerate}
	
	\item For each of the surfaces (a)-(g), determine if the proposed descriptions of the level curves are correct. If not, give a correct descriptor. \textit{[Note: consider a point as a circle/ellipse of radius 0]}
		\begin{enumerate}
			\item $z=2x^2-3y^2$; Level curves are concentric ellipses. 
			\item $z=x^2+y^2$; Level curves are concentric circles
			\item $z=\dfrac{1}{x+y}$; Level curves are lines, whenever $x \neq -y$. 
			\item $z=2x+3y$; Level curves are parallel planes. 
			\item $z=\sqrt{25-x^2-y^2}$; Level curves are concentric circles, but only if $z> 5$ or $z<-5$
			\item $z=\sqrt{x^2+y^2}$; Level curves are concentric circles, but only if $z\geq 0$. 
			\item $z=xy$; Level curves are hyperbolas.
		\end{enumerate}
\end{enumerate}

\subsection*{Applications}
\begin{enumerate}[resume]
    \item Multivariable functions are often used in economic models to describe how one should price an asset, or how to determine the utility of a product. For example, consider a \textit{utility function} $u(x,y,z)$, where $x,y,z$ represent three independent properties of an object (eg., price, quantity, quality), and $u$ tells you how much you value that item. In this context, what economic significance do the level surfaces $u(x,y,z) = C$ have (assume $C$ is a constant). Give a example of how this phenomenon might manifest in your day-to-day life. 
\end{enumerate}
\subsection*{Extensions}
\begin{enumerate}[resume]
    \item Find an equation for the level surface of the function $f(x,y,z)=\sqrt{x^2+y^2+z^2}$ passing through $(1,1,1)$. Sketch a plot of this level surface in $\mathbb{R}^3$.
    \item Let $f(x,y) = (x-y)^2$. Determine the equations and shapes of the cross-sections when $x = 0$, $y = 0$, and $x = y,$ and describe the level curves. Use this information to produce a sketch of the graph of the surface.  Confirm your sketch using a 3d graphing utility.
\end{enumerate}}{}

\iftoggle{answers}{
\begin{center}
	{\large \textbf{Math 2551 Worksheet Answers: Multivariable Functions}}
\end{center}

\begin{enumerate}	
	\item Find and sketch the domain for each function.
	\begin{enumerate}
		\item $\{ (x,y)\mid x-y\geq 1 \}$
		
		\item $\{ (x,y) \mid x\geq 4, |y|\geq 1 \} \cup \{(x,y) \mid x<4, |y|<1 \}$
		
		\item $\{ (x,y) \mid 4x^2-1\leq y\leq 4x^2+1 \}$
		
		\item All of $\R^2$ except the circle $x^2+y^2=4$
		
		\item All of the disk $x^2+y^2< 4$ except the circle $x^2+y^2=3$.
	\end{enumerate}
	
	\item	
	\begin{enumerate}
		\item  a collection of concentric ellipses
		\item a collection of unequally spaced concentric circles
		\item a collection of unequally spaced parallel lines
		\item a collection of equally spaced parallel lines
		\item a collection of unequally spaced concentric circles
		\item a collection of equally spaced concentric circles
		\item two straight lines and a collection of hyperbolas
	\end{enumerate}
	
	\item The plane $4x+y+3=0$, except for those points with $x+y=-1$.
	
	\item The sphere $3=x^2+y^2+z^2$
	
	\item When $x=0$, the cross-section is the parabola $z=y^2$. 
	
	When $y=0$, the cross-section is the parabola $z=x^2$.
	
	When $x=y$, the cross-section is the line $z=0$.
	
	The level curves are pairs of parallel lines $y=x\pm\sqrt{k}$.
\end{enumerate}
}{}
\iftoggle{solutions}
{
Solutions go here in the same format.
}{}

% 
\fancyhead[C]{Section 14.2}
	\fancyhead[R]{\dayseven}
\iftoggle{questions}{	
\begin{center}{\large \textbf{Math 2551 Worksheet: Limits and Continuity}}
\end{center}


\begin{enumerate}	
	\item  Let $f(x,y)=\left(\dfrac{1}{x}+\dfrac{1}{y}\right)^2$. Find $\displaystyle \lim_{(x,y) \to (2,-3)} f(x,y)$ or show it does not exist.
	
	\item Let $f(x,y)= \dfrac{x-2y}{x^3-8y^3}$.  Find $\displaystyle \lim_{(x,y) \to (2,1)} f(x,y)$ or show it does not exist.
	
	\item Let $f(x,y) = \dfrac{\sqrt{2x-y}-2}{2x-y-4}$.  Find $\displaystyle \lim_{(x,y) \to (2,0)} f(x,y)$ or show it does not exist.
	
	\item Let $f(x,y)= \dfrac{y^2}{x^2+y^2}$.  Find $\displaystyle \lim_{(x,y) \to (0,0)} f(x,y)$ or show it does not exist.
	
	\item At what points $(x,y)$ in the plane is $f(x,y)=\cos\left(\dfrac{1}{xy}\right)$ continuous?  
	\item At what points $(x,y,z)$ is $h(x,y,z)= \dfrac{1}{1-\ln{(x^2+y^2+z^2)}}$ continuous?
	
\end{enumerate}
}{}

\iftoggle{answers}{
\begin{center}{\large \textbf{Math 2551 Worksheet Answers: Limits and Continuity}}
	\end{center}
\begin{enumerate}	
	\item  $\dfrac{1}{36}$
	
	\item $\dfrac{1}{12}$
	
	\item $\dfrac{1}{4}$
	
	\item Does not exist.
	
	\item $f$ is continuous on its entire domain: all $(x,y)$ such that neither $x=0$ nor $y=0$.
	
	\item $f$ is continuous on its entire domain: all $(x,y,z)$ except the sphere $x^2+y^2+z^2=e$.
\end{enumerate}
}{}
\iftoggle{solutions}
{
Solutions go here in the same format.
}{}
% 
\fancyhead[C]{Sections 12.1-6, 13.1-4, 14.1-2}
\fancyhead[R]{\dayeight}
	
\iftoggle{questions}{\begin{center}{\large \textbf{Math 2551 Worksheet 8 - Review for Exam 1}}
\end{center}


\begin{enumerate}	
	\item Set up the integral to find the arc length of the curve $y=e^x$ from the point $(0,1)$ to the point $(1,e)$.  Focus on finding a parameterization, and on what values of $t$ give these two points.  Is this an integral you would want to compute? Why or why not?
	
	\item Parameterize the line tangent to the curve 
	\[ \br(t)=\langle \cos^2(t),\sin(t)\cos(t),\cos(t)\rangle \]
	at the point where $t=\pi/2$.
	
	\item Compute the unit tangent vector $\bT(t)$ and the unit normal vector $\bN(t)$ to the circle 
	\[\br(t)=\langle 2\cos(t),2\sin(t)\rangle. \]
	Before checking, should the normal vector be pointing into or out of the circle? Why?
	
	\item We have seen that the curvature of a circle with radius $a$ is $1/a$.  Thinking about the geometry of a helix with radius $a$, do you think its curvature will be greater than or less than $1/a$?  Why?  Compute the curvature using the parameterization 
	\[\br(t)=\langle a\cos(t), t, a\sin(t) \rangle \]
	to confirm or challenge your intuition.
	
	\item The function $\mathbf{\ell}(t)$ below describes a line.  There is a particular plane that $\mathbf{\ell}(t)$ is normal to at the point $t=0$.  Find an equation of this plane.
	\[\mathbf{\ell}(t)=\langle 3-3t,2+t,-2t\rangle. \]
	
	Where does this line intersect the different plane $3x-y+2z=-7$?
	
	\item Find and sketch the domain of each of the following functions of two variables:
	\begin{enumerate}
		\item $\sqrt{9-x^2}+\sqrt{y^2-4}$
		\item $\arcsin(x^2+y^2-2)$
		\item $\sqrt{16-x^2-4y^2}$
	\end{enumerate}

	\item Solve the differential equation below, together with its given initial conditions.  Remember that this means finding all functions $\br(t)$ which satisfy the given equations.
	
		\[ \br''(t)=2\bi+6t\bj+\dfrac{1}{2\sqrt{t}}\bk, \quad \br'(1)=2\bi+3\bj+\bk, \quad \br(1)=\bi+\bj \]
		
		
		
		\item Let $f(x,y)=(x^2-y^2)/(x^2+y^2)$ for $(x,y)\neq (0,0)$. Is it possible to define $f(0,0)$ in a way that makes $f$ continuous at the origin? Why?
	
\end{enumerate}
}
\iftoggle{answers}{

\begin{center}{\large \textbf{Math 2551 Worksheet 8 Answers - Review for Exam 1}}
\end{center}


\begin{enumerate}	
	\item $\int_0^1 \sqrt{1+e^{2t}}\ dt$
	
	\item $\mathbf{\ell}(s)=\langle 0,-s,-s\rangle$
	
	\item $\bT(t)=\langle -\sin(t),\cos(t)\rangle$
	
	$\bN(t)=\langle -\cos(t),-\sin(t)\rangle$
	
	Into
	
	\item $\kappa=\dfrac{a}{1+a^2}$
	
	\item $-3(x-3)+(y-2)-2z=0$ \\
	Intersection point is $(0,3,-2)$, when $t=1$.
	
	\item \begin{enumerate}
		\item $\{(x,y)\mid |x|\leq 3,|y|\geq 2 \}$
		\item $1\leq x^2+y^2\leq 3$
		\item $\dfrac{x^2}{16}+\dfrac{y^2}{4}\leq 1$
	\end{enumerate}

	\item $\br(t)=t^2\bi+t^3\bj+\frac{2}{3}(t^{3/2}-1)\bk$
	\item No, because the limit of $f$ as $(x,y)\to(0,0)$ does not exist.
\end{enumerate}
}{}
\iftoggle{solutions}
{
Solutions go here in the same format.
}{}

% 
\fancyhead[C]{Section 14.3}
\fancyhead[R]{\daynine}
\iftoggle{questions}
{\begin{center}{\large \section*{\centering Chapter 14.3: Partial Derivatives}}
\end{center}
\subsection*{Mechanics}
\begin{enumerate}
    \item Find all first and second partial derivatives for $f(x,y)=e^x+x\ln (y).$
    \item Find $f_x$, $f_y$, $f_z$, and $f_{xzz}$ for the function $f(x,y,z)=x\sin(yz)$.
 \item Find the total derivative $Df$ at the given point for each function below. Remember that $Df$ is the matrix of (partial) derivatives of the function and if $f$ is a function from $\R^n$ to $\R^m$ then $Df$ is a $m\times n$ matrix.
	\begin{enumerate}
		\item $f(x)=2x^3+7$ at $x=2$.
		\item $\mathbf{f}(t)=\langle 2\cos(t), 2 \sin(t), t\rangle$ at $t=\pi/2$.
		\item $f(x,y)=\sqrt{y-x}$ at $(x,y)=(1,2)$.
		\item $f(x,y,z)=e^{2y-x}+z^2+4$ at $(x,y,z)=(1,2,3)$.
		\item $\mathbf{f}(s,t) = \langle 2s+3t, t-s\rangle$ at $(s,t)=(1,1)$.  
	
		\textbf{Note:} The graph of this function is a surface (in this case all of $\R^2$) parameterized by two variables just like the graph of the function in (b) is a curve parameterized by one variable - we'll see these more later!  Another way of thinking about this is that this is a \textit{change of variables} for $\R^2$ between the system of coordinates $(s,t)$ and $(x,y)$.
	\end{enumerate}
    
\end{enumerate}
\subsection*{Applications}
\begin{enumerate}[resume]
    \item The speed of sound $C$ traveling through ocean water is a function of 
	temperature, salinity, and depth.  It may be modeled by the function	
	\[C(T,S,D)=1450 +4.5T-0.05T^2+0.0003T^3+(1.5-0.01T)(S-35)+0.015D, \]
	
	where $C$ is the speed of sound in meters/second, $T$ is the temprature in degrees Celsius, $S$ is the salinity in grams/liter of water, and $D$ is the depth below the ocean surface in meters.
	
	\begin{enumerate}
		\item State the units in which each of the partial derivatives $C_T,C_S,$ and $C_D$ are expressed and explain the physical meaning of each.
		
		\item Find the partial derivatives $C_T, C_S,$ and $C_D$.
		
		\item Evaluate each of the three partial derivatives at the point where $T=10, S=35$, and $D=100$.  What does the sign of each partial derivative tell us about the behavior of the function $C$ at the point $(10,35,100)$?
	\end{enumerate}
    \pagebreak 
    
    \item Recall from last week's worksheet that a utility function is a multivariable function $u(x,y,z)$, where $x,y,z$ represent three independent properties of an object (eg., price, quantity, quality), and $u$ tells you how much you value that item. The \textit{marginal utility functions} are the partial derivatives $u_x,u_y$ and $u_z$. What is the economic interpretation of the marginal utilities? 
\end{enumerate}
\subsection*{Extensions}
\begin{enumerate}[resume]
	
	\item Below is a contour plot for a function $f(x,y)$, with values for some of the contours (level curves) indicated on the \textit{left} of the figure.
	
	\begin{minipage}{0.6\textwidth}
		\begin{enumerate}
			\item Find the sign of the partial derivatives \\
            $f_x(-2,-1)$ and $f_y(-2,-1)$.
			\item At the point $(0,-1/2)$, which is larger? $f_x$ or $f_y$?
			\item Find all $(x,y)$ where $f_x(x,y)=0$.
			\item Locate, if possible, one point $(x,y)$ where\\ $f_x(x,y)<0$.
		\end{enumerate}
	\end{minipage}
	\begin{minipage}{0.4\textwidth}
		\includegraphics[scale=0.6]{contour_14_3.png}
	\end{minipage}
	

	

%	\item Find $f_x$ and $f_y$ for:
%	\begin{enumerate}
%		\item $f(x,y)= x^3y^2+5y^2-x+7$
%		
%		\item $f(x,y) = e^{x^2y^3} \sqrt{x^2+1}$
%		
%		\item $f(x,y) = \cos(xy^2)+\sin(x)$
%	\end{enumerate}

	\item The fifth-order partial derivative $\partial^5f/\partial x^2\partial y^3$ is zero for each of the following functions.  To show this as quickly as possible, which variable would you differentiate with respect to first: $x$ or $y$?
	
	Try to answer without writing anything down.  Why did you make the choice you did?
	
	\begin{enumerate}
		\item $f(x,y)=y^2x^4e^x+2$
		
		\item $f(x,y)=y^2+y(\sin(x)-x^4)$
		
		\item $f(x,y)=x^2+5xy+\sin(x)+7e^x$
		
		\item $f(x,y)=xe^{y^/2}$
	\end{enumerate}
    \item Let $A$ be any $2\times 2$ matrix, and let $\mathbf{f}: \mathbb{R}^2\to\mathbb{R}^2$ be given by $\mathbf{f}(\mathbf{x}) = A\mathbf{x}$. Compute the total derivative $D\mathbf{f}$. What do you notice? What familiar family of functions from Calc 1 does this remind you of? Can you generalize this result? 
\end{enumerate}
}{}

\iftoggle{answers}{
\begin{center}{\large \textbf{Math 2551 Worksheet Answers: Partial Derivatives}}
\end{center}
\begin{enumerate}	
	\item 
	\begin{enumerate}
		\item $f_x(-2,-1)\approx 0.75$
		\item $f_y(-2,-1)\approx 1.5$
		\item There are several possible points (these are places where the tangent to a contour is horizontal): $(0,-.5),(-.5,-1.25)$, etc.
		\item Again, there are many possible points; any point on the 4, 5, 6 contours in quadrant IV will work.
	\end{enumerate}

	\item \begin{enumerate}
		\item $C_T$: (meters/second)/ degree Celsius - this gives the change in speed for each one degree C of temperature increase.
		$C_S$ (meters/second)/(grams/liter) - this gives the change in speed for each one gram/liter increase in salinity
		$C_D$: (meters/second)/meter - this gives the change in speed for each one meter increase in depth below the surface
		
		\item $C_T= 4.5-0.1T+0.0009T^2-0.01(S-35)$
		$C_S=1.5-0.01T$
		$C_D=0.015$
		
		\item At $(T,S,D)=(10,35,100)$, we have $C_T=3.59, C_S=1.4, C_D=0.015$.  This tells us that if we increase the temperature, salinity, or depth from these conditions the speed of sound will increase as well.
	\end{enumerate}
%	\item 	\begin{enumerate}
%		\item $f_x=3x^2y^2-1, f_y=2x^3y+10y$
%		
%		\item $f_x=2xy^3e^{x^2y^3}\sqrt{x^2+1}+\dfrac{x e^{x^2y^3}}{\sqrt{x^2+1}}, f_y=3x^2y^2 e^{x^2y^3}\sqrt{x^2+1}$
%		
%		\item $f_x=-y^2\sin(xy^2)+\cos(x), f_y=-2xy\sin(xy^2)$
%	\end{enumerate}
	
	\item $f_{xx}=e^x, f_{xy}=f_{yx}=\dfrac{1}{y}, f_{yy}=-\dfrac{x}{y^2}$
	
	\item $f_x=\sin(yz), f_y=xz\cos(yz), f_z=xy\cos(yz), f_{xzz}=-y^2\sin(yz)$
	
	\item Find the total derivative $Df$ at the given point for each function below. Remember that $Df$ is the matrix of (partial) derivatives of the function and if $f$ is a function from $\R^n$ to $\R^m$ then $Df$ is a $m\times n$ matrix.
	\begin{enumerate}
		\item $Df(2)=f'(2)=[24]$
		\item $D\mathbf{f}(\pi/2)=\mathbf{f}'(\pi/2)=\begin{bmatrix}
			-2 \\ 0 \\ 1
		\end{bmatrix}$
		\item $Df(1,2)=\begin{bmatrix}
			-1/2 & 1/2
		\end{bmatrix}$
		\item $Df(1,2,3)=\begin{bmatrix}
			-e^3 & 2e^3 & 6
		\end{bmatrix}$
		\item $D\mathbf{f}(1,1) = \begin{bmatrix}
			2 & 3 \\
			-1 &  1
		\end{bmatrix}  $
		
	\end{enumerate}
	\item Note this does not have a definitive right answer - some differences may arise and that's good! Discuss!
	
	\begin{enumerate}
		\item First $y$ since $\partial^3 f/\partial y^3=0$ and the $y$-partial derivatives are easier
		
		\item First $y$, since $\partial^3 f/\partial y^3=0$
		
		\item First $y$, since $\partial^2 f/\partial y^2=0$
		
		\item First $x$, since $\partial^2 f/\partial x^2=0$ and the $x$-partial derivatives are easier.
	\end{enumerate}
	
	A common theme is to work with the variable with lower powers/simpler expressions first when taking mixed partials.
	
\end{enumerate}
}{}
\iftoggle{solutions}
{
Solutions go here in the same format.
}{}

% 
\fancyhead[C]{Section 14.4}
\fancyhead[R]{\dayten}

\iftoggle{questions}{
\begin{center}{\large \textbf{Math 2551 Worksheet: Chain Rule}}
\end{center}

\begin{enumerate}
	\item An object travels along a path on a surface.  The exact path and surface are not known, but at time $t=t_0$ it is known that
	\[ \pdev{z}{x}=5, \quad \pdev{z}{y}=-2, \quad \dfrac{dx}{dt}=3, \quad \dfrac{dy}{dt}=7. \]
	Use this information to determine the rate of change of the height $z$ of the object with respect to time $t$ at $t=t_0$.
	
	\item Find the values of $t$ where $\dfrac{dz}{dt}=0$ if $z=3x+4y$, $x=t^2$, and $y=2t$.
	
	\item Let $w= xy + yz + zx$, where $x= r \cos \theta, \ \ y= r \sin \theta, \ \  z= r \theta.$ 	Find $\dfrac{\partial w}{\partial r}$ and $\dfrac{\partial w}{\partial \theta}$ when $r=2$ and $\theta = \dfrac{\pi}{2}$.
	
	\item Suppose we have a differentiable function $w=g(x,y)$ and $x$ and $y$ are differentiable functions of $t$ and we know the following information.
	\[g(1,0)=1,\ g_x(1,0)=-2,\ g_y(1,0)=2,\ g(-1,2)=3,\ g_x(-1,2)=1,\ g_y(-1,2)=-2,\]
	\[ x(2)=1,\ y(2)=0,\ x(1)=1,\ y(1)=3,\ x'(2)=4,\ y'(2)=-1,\ x'(1)=0,\ y'(1)=2 \]
	
	If possible, find $\dfrac{dw}{dt}(1)$ and $\dfrac{dw}{dt}(2)$ or explain why the given information is not enough to do so.  Which of these pieces of information would you not use at all to compute either value?

\end{enumerate}
}{}

\iftoggle{answers}{
\begin{center}{\large \textbf{Math 2551 Worksheet Answers: Chain Rule}}
\end{center}

\begin{enumerate}
	\item $\dfrac{dz}{dt}(t_0)=1$
	
	\item $t=-4/3$
	
	\item $\dfrac{\partial w}{\partial r}(2,\pi/2)=2\pi$ \qquad $\dfrac{\partial w}{\partial \theta}(2,\pi/2)=-2\pi$
	
	\item $\dfrac{dw}{dt}(2)=-10$.  $\dfrac{dw}{dt}(1)$ cannot be computed from the given information because we do not know the values of $g_x$ or $g_y$ at $(x(1),y(1))=(1,3)$.  We do not use the values of $g(1,0),g(-1,2),g_x(-1,2),g_y(-1,2)$.
\end{enumerate}

}{}
\iftoggle{solutions}
{
Solutions go here in the same format.
}{}

 
\fancyhead[C]{Section 14.5}
\fancyhead[R]{\dayeleven}

\iftoggle{questions}{
\begin{center}{\large \textbf{Math 2551 Worksheet: Gradient and Directional Derivatives}}
\end{center}

\begin{enumerate}
	\item Use the contour diagram of the differentiable function f given below to decide if
	the specified directional derivative is positive, negative, or approximately zero.
	
	\begin{minipage}{0.3\linewidth}
		\includegraphics[scale=0.7]{contour_14_5.png}
	\end{minipage}
	\begin{minipage}{0.6\linewidth}	
		\begin{enumerate}
			\item At the point $(-2,2)$ in the direction $\bi$
			\item At the point $(0,-2)$ in the direction $\bj$
			\item At the point $(-1,1)$ in the direction $\bi +\bj$
			\item At the point $(-1,1)$ in the direction $-\bi+\bj$
			\item At the point $(0,-2)$ in the direction $\bi - 2\bj$
		\end{enumerate}
	\end{minipage}
	
	\item Let $f(x,y)=xy$.  Sketch the curve $f(x,y)= -4$ together with $\nabla f(2,-2)$ and the tangent line at $(2,-2)$. Then, find an equation for the tangent line.  What do you notice?
	
	
	\item Find the derivative of $g(x,y)= \dfrac{x-y}{xy+2}$ at $(1,-1)$ in the direction of $\langle 12, 5\rangle$.

	
	\item Suppose you are climbing a hill whose shape is given by the equation
	\[ z = 1000 - 0.005x^2-0.01y^2, \]
	where $x, y,$ and $z$ are measured in meters, and you are standing at a point with
	coordinates $(60, 40, 966)$. The positive $x$-axis points east and the positive $y$-axis
	points north.
	\begin{enumerate}
		\item If you walk due south, will you start to ascend or descend? At what rate?
		\item If you walk northwest, will you start to ascend or descend? At what rate?
		\item In which direction is the slope largest? What is the rate of ascent in that direction?
	\end{enumerate} 
	\item Let $f(x,y)=-x^2y+xy^2+xy$ and $P=(2,1)$.
	\begin{enumerate}
		\item Find the direction of maximal increase of $f$ at $P$.
		\item What is the maximum rate of change of $f$ at $P$?
		\item Find the direction of maximal decrease of $f$ at $P$.
		\item Find a direction $\bu$ such that $D_{\bu}f(P)=0$ (note this forces $\bu$ to be a unit vector!).
	\end{enumerate}
\end{enumerate}
}{}

\iftoggle{answers}{
\begin{center}{\large \textbf{Math 2551 Worksheet Answers:Gradient and Directional Derivatives}}
\end{center}

\begin{enumerate}

		\item 	\begin{enumerate}
			\item Negative
			\item Negative
			\item Approximately zero
			\item Positive
			\item Positive
		\end{enumerate}
		\item Tangent line: $-2(x-2)+2(y+2)=0$\\
		
	\includegraphics{14_5_sketch_soln.png}
		
		\item $D_{\bu}g(1,-1)=\dfrac{21}{13}$
		
		\item \begin{enumerate}
			\item Ascend at a rate of 0.8 vertical meters per horizontal meter
			\item Descend at a rate of $\sqrt{2}/10$ vertical meters per horizontal meter
			\item $\langle -0.6, -0.8\rangle$ is the direction of largest slope with rate of ascent 1 vertical meter per horizontal meter.
		\end{enumerate}
		\item 
		\begin{enumerate}
			\item $\langle -1/\sqrt{2},1/\sqrt{2} \rangle$
			\item $2\sqrt{2}$
			\item $\langle 1/\sqrt{2},-1/\sqrt{2} \rangle$
			\item $\langle 1/\sqrt{2},1/\sqrt{2} \rangle$
		\end{enumerate}
\end{enumerate}

}{}
\iftoggle{solutions}
{
Solutions go here in the same format.
}{}

% \fancyhead[C]{Section 14.6}
\fancyhead[R]{\daytwelve}

\section*{\centering Chapter 14.6: Linearization and Tangent Planes}

\textbf{D2: Tangent Planes and Linear Approximations.} I can find equations for tangent planes to surfaces and linear approximations of functions at a given point and apply these to solve problems.

\subsection*{Mechanics}
\begin{enumerate}	
    \item %14.6 # 30b
    \question{Find the linearization of $f(x,y)=e^{2y-x}$ at $(1,2)$. Without doing any more calculations, find an equation of the tangent plane of the surface $f(x,y)$ at $(1,2)$.}
    {% answer goes here
    $L(x,y)=e^3-e^3(x-1)+2e^3(y-2)$
    
    tangent plane: $z=e^3-e^3(x-1)+2e^3(y-2)$
    }
    {% solution goes here
    }
     
	
	\item \question{Find the linearization of $f(x,y,z)=\arctan(xyz)$ at $(1,1,0)$.}
    {% answer goes here
    $L(x,y,z)=z$
    }
    {% solution goes here
    } 
	\item \question{Use the linearization to approximate $f(2.95, 7.1)$ for the function $f(x,y)=\sqrt{x^2+y}$, knowing that $f(3,7)=4$.	}
    {% answer goes here
    $f(2.95, 7.1)\approx 4-1/40$
    }
    {% solution goes here
    }
    \item \question{Find an equation of the tangent plane to the unit sphere $x^2+y^2+z^2=1$ at the point $\left(\dfrac{1}{\sqrt{2}},\dfrac{1}{2},\dfrac{1}{2}\right)$.}
    {% answer goes here
        $\sqrt{2}\left( x-\dfrac{1}{\sqrt{2}}\right)+\left(y-\dfrac{1}{2}\right)+\left(z-\dfrac{1}{2}\right)=0$
    }
    {% solution goes here
    }
\end{enumerate} 
\subsection*{Applications}
\begin{enumerate}[resume]
    \item \question{Suppose you are shining a flashlight on a smooth surface. The \emph{angle of incidence}, $\theta_i$ is the angle at which a light beam hits the surface, measured with respect to the surface normal (i.e., the normal vector to the tangent plane at point of contact). 
    The \emph{angle of reflection} $\theta_r$ is the angle of the reflected light beam measured with respect to the surface normal. 
    The \emph{law of reflection} states that in a vacuum, we must have $\theta_i=\theta_r$. 
    Draw a labeled picture to convince yourself that this is reasonable.
    
    Now, consider the paraboloid $z = x^2+y^2$, and a light ray traveling along the path $\mathbf{r}(t)= (-2,-3,2)t+(3,4,0)$. 
    Compute the angle of reflection at the point $(1,1,2)$ 
    [\emph{Hint: How can one find the angle between two vectors?}]. }
    {% answer goes here
     $\theta_r = \arccos(\frac{4}{\sqrt{17}})\approx 0.245\ \text{rad}$
    }
    {% solution goes here
    $\mathbf{n}=(2,2,-1),\theta_i=\arccos(\frac{4}{\sqrt{17}})\approx 0.245\ \text{rad}$
    } 
\end{enumerate}
\subsection*{Extensions}
\begin{enumerate}[resume]	
    \item \question{Use software to graph the function $z=x^{1/3}y^{1/3}$.  
    Examine the graph at $(0,0)$ - does it look like the function has a tangent plane there? 
    Use this to deduce a necessary condition for a function $f(x,y)$ to have a tangent plane. }
    {% answer goes here
    No, there are two different tangent planes.  The function cannot have a cusp at the point with the tangent plane.
    }
    {% solution goes here
    } 
\end{enumerate}
% 
\fancyhead[C]{Section 14.7}
	\fancyhead[R]{\daytwelve}
\iftoggle{questions}{
\begin{center}{\large \textbf{Math 2551 Worksheet: Optimization I}}
\end{center}


\begin{enumerate}
	
	\item Find all the local maxima, local minima, and saddle points of $f(x,y) = e^y(x^2-y^2)$.
	
	\item Find and classify all critical points for the function $x^3+3xy+y^3$.
	
	\item  Can you conclude anything about $f(a, b)$, if $f$ and its first and second partial
	derivatives are continuous around the critical point $(a, b)$
	and $f_{xx}(a,b)$ and $f_{yy}(a,b)$ differ in sign? Justify your answer.
	
	\item In each case, the origin is a critical point of $f$ and $f_{xx}f_{yy}-(f_{xy})^2 = 0$ at the origin, so the Second Derivative Test fails at the origin.  Use some other method to determine whether the function $f$ has a maximum, a minimum, or neither at the origin.
	\begin{enumerate}
		\item $f(x,y)= x^2y^2$
		\item $f(x,y)= 1-x^2y^2$
		\item $f(x,y)= xy^2$
		\item $f(x,y)= x^3y^2$
		\item $f(x,y)= x^3y^3$
		\item $f(x,y)= x^4y^4$
	\end{enumerate}
	
\end{enumerate}
}{}

\iftoggle{answers}{
\begin{center}{\large \textbf{Math 2551 Worksheet Answers: Optimization I}}
\end{center}

\begin{enumerate}
	\item Saddle point at $(0,0)$ and local minimum at $(0,-2)$.
	
	\item Saddle point at $(0,0)$ and local maximum at $(-1,-1)$.
	
	\item Yes, this must be a saddle point because $f_{xx}(a,b)f_{yy}(a,b)<0$ so $\det(Hf)=f_{xx}(a,b)f_{yy}(a,b)-f_{xy}^2(a,b)<0$.
	
	\item
	\begin{enumerate}
		\item Minimum is $0$ at $(0,0)$ since $f(x,y)>0$ for all other $(x,y)$.
		
		\item Maximum is $1$ at $(0,0)$ since $f(x,y)<1$ for all other $(x,y)$.
		
		\item Neither since $f(x,y)<0$ for $x<0$ and $f(x,y)>0$ for $x>0$.
		
		\item Neither since $f(x,y)<0$ for $x<0$ and $f(x,y)>0$ for $x>0$.
		
		\item Neither since $f(x,y)<0$ for $x<0$ and $y>0$, but $f(x,y)>0$ for $x>0$ and $y>0$.
		
		\item Minimum is $0$ at $(0,0)$ since $f(x,y)>0$ for all other $(x,y)$.
	\end{enumerate}

\end{enumerate}
}{}
\iftoggle{solutions}
{
Solutions go here in the same format.
}{}

% 
\fancyhead[C]{Section 14.7}
	\fancyhead[R]{\daythirteen}
\iftoggle{questions}{
\begin{center}{\large \textbf{Math 2551 Worksheet: Optimization II}}
\end{center}


\begin{enumerate}
	\item Find the absolute maxima and minima of the function $f(x,y)=x^2-xy+y^2+1$ on the closed triangular plate bounded by lines $x=0$, $y=4$, $y=x$ in the first quadrant.
	
	\item Among all rectangular boxes of volume $27$ cm$^3$, what are the dimensions of the box with the smallest surface area?  What is the smallest possible surface area?  (assume this occurs at a local min of the surface area function)
	
\end{enumerate}
}{}

\iftoggle{answers}{
\begin{center}{\large \textbf{Math 2551 Worksheet Answers: Optimization II}}
\end{center}

\begin{enumerate}

	\item The absolute maximum is $17$, achieved at $(0,4)$ and $(4,4)$, and the absolute minimum is $1$, achieved at $(0,0)$.
	
	\item The dimensions are $3 \times 3 \times 3$ and the surface area is $54$.
	
\end{enumerate}
}{}
\iftoggle{solutions}
{
Solutions go here in the same format.
}{}

% 
\fancyhead[C]{Section 14.8}
\fancyhead[R]{\daythirteen}

\section*{\centering Chapter 14.8: Lagrange Multipliers}

\textbf{D3: Optimization.} I can locate and classify critical points of functions of two variables. I can find absolute maxima and minima on closed bounded sets. I can use the method of Lagrange multipliers to maximize and minimize functions of two or three variables subject to constraints. I can interpret the results of my calculations to solve problems.

\subsection*{Mechanics}
\begin{enumerate}	
	\item \question{Find the extreme values of the function $f(x,y)=x^2+2y^2$ on the circle $x^2+y^2=1$.}
	%1, 2}
    {% answer goes here
    The extreme values are 1 and 2.
    }
    {% solution goes here
    } 
    \item \question{Find the extreme values of the function $f(x,y,z)=x^2+y^2+z^2$ subject to the constraint $x^4+y^4+z^4=1$.}
    {% answer goes here
    The extreme values are 1 and $\sqrt{3}$.
    }
    {% solution goes here
    } 

\end{enumerate}
\subsection*{Applications}
\begin{enumerate}[resume]
    \item \question{A rectangular box without a lid is to be made from 12 $m^2$ of cardboard. Find the maximum volume of such a box.}  %(2,2,1)
    {% answer goes here
    $V(2,2,1)=4$ cubic units
    }
    {% solution goes here
    } 
    \item \question{A niche restaurant in midtown Atlanta serves only garlic bread (denoted by $g$) and bunches of kale (denoted by $k$). The cost of producing these goods is given by the function $C(g,k) = 5g^2+2gk+3k^2+10$. Assuming that the total amount of items to be produced is 40, compute the minimal production cost.}
    {% answer goes here
    $ C_{min}=\frac{11230}{3} \approx \$3743.33$.
    }
    {% solution goes here
    } 
    \item \question{The height of a mountain is given by $h(x,y)= 300-(3x^2+4xy+3y^2)$. Compute the height of the lowest point on the mountain within $\sqrt{200}$ units of the origin. Where might this point occur? Are there multiple points where this occurs? \textit{[Hint: First, justify why none of $x,y,\lambda$ can be zero. Then solve for $\lambda$.}]}
    {% answer goes here
    The two minimums occur at $(x,y)= (\pm 10,\pm 10)$, where the height is -700 units. Warning: The two other solutions to the Lagrange system are $(x,y)=(\pm 10,\mp 10)$ and they are maximums at height 100.
    }
    {A routine check shows that the only critical point is at $(0,0)$, which is not a minimum by the second derivative test. Therefore, we check on the boundary $x^2+y^2 = 200$. The Lagrange system looks like 
    \begin{align*}
        -6x-4y &= 2\lambda x\\
        -4x-6y &= 2\lambda y\\
        x^2+y^2 &= 200
    \end{align*}
    Note that if either $x,y$ or $\lambda$ is zero, at least one of the three equations will be violated, hence we are free to divide at leisure. Solving for $\lambda$ in two different ways and equating gives us $x/y = y/x$. Plugging into the constraint equation yields that $x=\pm 10$ and $y=\pm 10$. A routine check distinguishes two of these as maximums and two as minimums.  
    } 
\end{enumerate}
\subsection*{Extensions}
\begin{enumerate}[resume]
    \item \question{The plane $x+y+2z=2$ intersects the paraboloid $z=x^2+y^2$ in an ellipse.  Find the points on this ellipse that are nearest to and farthest from the origin. \textit{[Hint: It may be helpful algebraically to work with the square of the distance to the origin.]}}
    {% answer goes here
    The closest point is $(\dfrac{1}{2},\dfrac{1}{2},\dfrac{1}{2})$ and the farthest point is $(-1,-1,2)$.
    }
    {% solution goes here
    } 

	
	\item \question{Find the maximum volume of a rectangular box that is inscribed in a sphere of radius $r$.}
    {% answer goes here
    $\dfrac{8r^3}{3\sqrt{3}}$
    }
    {% solution goes here
    } 
    
\end{enumerate}
% 
\fancyhead[C]{Section 15.1}
\fancyhead[R]{\dayfifteen}

\section*{\centering Chapter 15.1: Double Integrals on Rectangles}
\textbf{I1: Double \& Triple Integrals.} I can set up double and triple integrals as iterated integrals over any region. I can sketch regions based on a given iterated integral.\\\\
\textbf{I2: Iterated Integrals.} I can compute iterated integrals of two and three variable functions, including applying Fubini's Theorem to change the order of integration of an iterated integral.

\subsection*{Mechanics}
\begin{enumerate}
    \item \question{Compute $\displaystyle \iint_R (xy-3xy^2) \ dA$, where  $R$ is the square $0 \leq x \leq 2,1 \leq y \leq 2$.}{-11
    }
    {% solution goes here
    } 
    \item \question{Use Fubini's Theorem to evaluate the integral
	\begin{equation*}
	    \int_0^1 \int_0^3 xe^{xy}\ dx\ dy 
	\end{equation*} 
    Why was it a good idea to exchange the order of integration?}{% answer goes here
    }
    {% solution goes here
    } 
    \item \question{Find the volume of the region bounded above by the paraboloid $z = 16 - x^2 - y^2$ and below by the square 
$R: 0 \leq x \leq 2, 0 \leq y \leq 2$.}{% answer goes here
    }
    {% solution goes here
    } 
\end{enumerate}
\subsection*{Extensions}
\begin{enumerate}[resume]
\item \question{Evaluate the double integral $\iint_R (4-2y)\ dA$, where $R=[0,1]\times[0,1]$ \textbf{without integrating} by identifying it as the volume of a solid \textit{[Hint: It is a prism cut by some plane.]}}{% answer goes here
    }
    {% solution goes here
    } 

\item \question{The integral $\iint_R \sqrt{9-y^2}$, where $R=[0,4]\times[0,2]$, represents the volume of a solid.  Sketch the solid.}{% answer goes here
    }
    {% solution goes here
    } 
	
	\item \question{This problem explores a failure of Fubini's theorem. Consider the two iterated integrals, which differ by swapping the order of integration:
	\begin{equation*}
	\int_0^1 \int_0^1 \frac{x^2-y^2}{(x^2+y^2)^2}\ dy\ dx \quad \textrm{ and }\quad \int_0^1 \int_0^1 \frac{x^2-y^2}{(x^2+y^2)^2}\ dx\ dy \end{equation*}
    Use the fact that 
    \begin{equation*}
        \frac{\partial}{\partial y} \left(\frac{y}{x^2+y^2}\right)=\frac{\partial}{\partial x} \left(\frac{-x}{x^2+y^2}\right)= \frac{x^2-y^2}{(x^2+y^2)^2}
    \end{equation*}
    and that $\frac{d}{dt}\arctan(t) = 1/(1+t^2)$ to show that the iterated integrals are different. Why does Fubini's theorem fail? }
    {% answer goes here
    }
    {% solution goes here
    } 
	
\end{enumerate}
{}

\iftoggle{answers}{
\begin{center}{\large \textbf{Math 2551 Worksheet Answers: Double Integrals on Rectangles}}
\end{center}

\begin{enumerate}
	\item 
	\includegraphics[scale=0.3]{ws_15_1_src.pdf}
	
	\item 
	\includegraphics[scale=0.25]{ws_15_2_src.pdf}
	
	\item $9\ln(2)$
	
	\item $160/3$ cubic units.
	
		\item $e^3-4$

	\item The integrals evaluate to $\pi/4$ and $-\pi/4$ respectively.  This does not violate Fubini's theorem because this function is not continuous on $[0,1]\times[0,1]$ (it has an asymptote at $(0,0)$)
\end{enumerate}
}{}
\iftoggle{solutions}
{
Solutions go here in the same format.
}{}

% \fancyhead[C]{Section 15.2}
	\fancyhead[R]{\dayfifteen}
	
\iftoggle{questions}{
\begin{center}{\large \textbf{Math 2551 Worksheet: Double Integrals on General Regions}}
\end{center}

\begin{enumerate}

\item Compute the integrated integral
\[\int_0^\pi \int_0^x x \sin (y)\ dy\ dx. \]


\item Decide, without calculation, if each of the integrals below are positive, negative, or zero. Let D be the region inside the unit circle centered at the origin. Let T, B, R, and L denote the regions enclosed by the top half, the bottom half, the right half, and the left half of unit circle, respectively.

\begin{multicols}{2}
	\begin{enumerate}
		\item $\iint_B (y^3+y^5)\ dA$
		\item $\iint_T (y^3+y^5)\ dA$
		\item $\iint_D (y^3+y^5)\ dA$
		\item $\iint_L (y^3+y^5)\ dA$
		\item $\iint_R (y^3+y^5)\ dA$
	\end{enumerate}
\end{multicols}

\item Write an iterated integral for $\iint_R 1 \ dA$ over the region $R$ using vertical cross-sections and horizontal cross-sections. 
\begin{enumerate}
	\item Bounded by $y=e^{-x}$, $y=1$, and $x=\ln 3$.
	
	\item Bounded by $y=x^2$ and $y=x+2$
\end{enumerate}

	\item Sketch the region of integration, reverse the order of integration, and evaluate the integral.
	\[\displaystyle \int_0^{\sqrt{\pi}} \int_y ^{\sqrt{\pi}} \cos(x^2) \ dx \ dy.\]
	\item Sketch the region of integration and evaluate the integral \[ \iint_R xy^2\ dA, \] where $R$ is enclosed by $x=0$ and $x=\sqrt{1-y^2}$.
	
	
	\item Find the volume of the solid bounded by the cylinder $y^2+z^2=4$ and the planes $x=2y$, $x=0$, $z=0$ in the first octant.
	
	\item The integral expression below gives the area of a region in the $xy$-plane. Sketch the region, labeling the bounding curves with their equations, and giving the coordinates of points where the curves intersect.  Then find the area of the region.
	\[ \int_0^2 \int_{x^2-4}^0\ dy\ dx + \int_0^4\int_{0}^{\sqrt{x}}\ dy\ dx \]
	
	
\end{enumerate}
}{}

\iftoggle{answers}{
\begin{center}{\large \textbf{Math 2551 Worksheet Answers: Double Integrals on General Regions}}
\end{center}

\begin{enumerate}
	
	\item $2+\frac{1}{2}\pi^2$
	
	\item \begin{enumerate}
		\item Negative
		\item Positive
		\item Zero
		\item Zero
		\item Zero
	\end{enumerate}
	
	
	\item \begin{enumerate}
		\item $\int_0^{\ln(3)} \int_{e^{-x}}^1\ dy\ dx$ and $\int_{1/3}^1 \int_{-\ln(y)}^{\ln(3)}\ dx\ dy$
		
		\item $\int_{-1}^2 \int_{x^2}^{x+2} \ dy\ dx$ and $\int_0^1 \int_{-\sqrt{y}}^{\sqrt{y}} \ dx\ dy + \int_1^4 \int_{y-2}^{\sqrt{y}}\ dx\ dy$
	\end{enumerate}
	\item $ \int_0^{\sqrt{\pi}} \int_0^{x} \cos(x^2) \ dy \ dx=0$
	\item $\dfrac{2}{15}$
	
	
	\item $16/3$
	
	\item $32/3$	
\end{enumerate}
}{}
\iftoggle{solutions}
{
Solutions go here in the same format.
}{}

% 
\fancyhead[C]{Section 15.3}
\fancyhead[R]{\dayfifteen}

\section*{\centering Chapter 15.3: More Double Integrals}
\textbf{I1: Double \& Triple Integrals.} I can set up double and triple integrals as iterated integrals over any region. I can sketch regions based on a given iterated integral.

\textbf{I2: Iterated Integrals.} I can compute iterated integrals of two and three variable functions, including applying Fubini's Theorem to change the order of integration of an iterated integral.

\subsection*{Mechanics}
\begin{enumerate}
    	\item \question{
         Consider the function $f(x,y)=xy$. Without performing any computations, do you think the average value of $f$ is larger over the square $0\leq x \leq 1, 0\leq y\leq 1$, or over the quarter circle $x^2+y^2\leq 1$ \textit{in the first quadrant}? Verify your guess by integrating
        }
        {% answer here
        
        }
        {% solution here
        
        }
       
        \item \question{
        A metal triangular plate with vertices $(0,0)$, $(2,0)$ and $(2,4)$ has temperature equal to $C(x,y) = xe^{xy}$ degrees Celsius. Compute the average temperature of the plate. \textit{[Hint: Choose a favourable order of integration.]}
        }
        {% answer here
        
        }
        {% solution here
        
        }
\end{enumerate}
\subsection*{Applications}
\begin{enumerate}[resume]
    \item \question{
        If $f(x,y)=100(y+1)$ represents the population density in people per square mile of a planar region on Earth, where $x$ and $y$ are measured in miles, find the number of people in the region bounded by the curves $x=y^2$ and $x=2y-y^2$.
        }
        {% answer here
        
        }
        {% solution here
        
        }
    \item \question{
        A rectangular can of Pringles chips may be modelled by the prism $0\leq x \leq 1$, $0\leq y \leq 1$ and $0\leq z\leq 5$. Assuming that the Pringles container is filled up with chips until the surface $z=x^2-y^2+3$, are there more chips or air in the can? \textit{[Note: The Pringles enthusiast may complain that their containers are supposed to be cylinders, not prisms. This nuance will be addressed when we work with polar coordinates.]}
        }
        {% answer here
        
        }
        {% solution here
        
        }
\end{enumerate}
\subsection*{Extensions}
\begin{enumerate}[resume]
	\item \question{
        An organism can be initially described as the solid with base $[0,1]\times [0,1]$ and height $z = e^{x+y}$. Suppose that the base of this organism grows at a rate of $t$ units per second in both the positive $x$ and positive $y$ directions. Compute the rate of change of the volume of the organism at $t=4$ seconds. \textit{[Hint: Set up an integral expression for the volume in terms of $t$, evaluate the integral, then differentiate with respect to $t$.]}
        }
        {% answer here
        
        }
        {% solution here
        
        }
	
\end{enumerate}
{}

\iftoggle{answers}{
\begin{center}{\large \textbf{Math 2551 Worksheet 16 Answers: Applications, Polar Double Integrals}}
\end{center}

\begin{enumerate}
	\item Answers will vary a bit through the estimation process
	
	4 subdivisions: $31.75 \leq T_{avg} \leq 52.5$\\
	16 subdivisions: $33.18  \leq T_{avg} \leq 50.06$\\
	25 subdivisions: $36.32  \leq T_{avg} \leq 49.8$\\
	
	Colorado is a rectangle, which makes it easy to subdivide. Wyoming would also work well.
	
	
	\item On square: $f_{avg}=\frac{1}{1}\cdot \frac{1}{4}=\frac{1}{4}$\\
	
	On quarter circle: $f_{avg}=\frac{1}{\pi/4}\cdot \frac{1}{8}=\frac{1}{2\pi}$ 
	\item 50 people
\end{enumerate}
}{}
\iftoggle{solutions}
{
Solutions go here in the same format.
}{}
% \fancyhead[C]{Section 15.4}
	\fancyhead[R]{\daysixteen}
\iftoggle{questions}{
\begin{center}{\large \textbf{Math 2551 Worksheet: Polar Double Integrals}}
\end{center}

\begin{enumerate}

	\item Evaluate $\displaystyle \iint_D y^2+3x\ dA$ where $D$ is the region in the 3rd quadrant between $x^2+y^2=1$ and $x^2+y^2=9$.
	
	\item Change the Cartesian integral
	\[
	\int_{0}^{2}\int_{-\sqrt{4-y^2}}^{\sqrt{4-y^2}} e^{-x^2-y^2} \ dx \ dy
	\]
	into an equivalent polar integral and evaluate the integral.
	
	
	\item Find the area of the region common to the interiors of the cardioids $r=1+\cos \theta$ and $r=1-\cos \theta$. 
	
	\begin{center}
		\pgfplotsset{width=7cm,compat=1.8}
		\usepgfplotslibrary{polar}
		\begin{tikzpicture}
			\begin{polaraxis}
				\addplot+[mark=none,domain=0:720,samples=600] 
				{1+cos(x)}; 
				\addplot+[mark=none,domain=0:720,samples=600] 
				{1-cos(x)};
				\addplot[samples=360, mark=none, fill=black!70!black, opacity=0.5, domain=90:270] {1+cos(x)};
				\addplot[samples=360, mark=none, fill=black!70!black, opacity=0.5, domain=0:90] {1-cos(x)};
				\addplot[samples=360, mark=none, fill=black!70!black, opacity=0.5, domain=270:360] {1-cos(x)};
				% equivalent to (x,{sin(..)cos(..)}), i.e.
				% the expression is the RADIUS
			\end{polaraxis}
		\end{tikzpicture}
	\end{center}
	
	\item Use a double integral to determine the volume of the solid that is inside the cylinder $x^2+y^2=16$, below $z=2x^2+2y^2$, and above the $xy$-plane.
	
	\item \textbf{Challenge:} An integral of great importance in statistics is the Gaussian integral $I=\Ds \int_0^\infty e^{-x^2}\ dx$.  The function $f(x)=e^{-x^2}$ has no elementary antiderivative, so this integral was hard to compute with the methods of single-variable calculus.
	
	Let $I^2=\Ds \left(\int_0^\infty e^{-x^2}\ dx\right)\left(\int_0^\infty e^{-y^2}\ dy\right)$.
	
	\begin{enumerate}
		\item Express $I^2$ as the limit of a double integral in polar coordinates with an appropriately chosen domain.  (Hint: As $R$ goes to infinity, what happens to a disk of radius $R$ centered at the origin?)
		
		\item Evaluate your double interal to compute the value of $I^2$.  Use this to find the value of the original Gaussian integral $I$.
	\end{enumerate}

	You can find some history of this integral \href{https://www.york.ac.uk/depts/maths/histstat/normal_history.pdf}{here}.
	
\end{enumerate}
}{}

\iftoggle{answers}{
\begin{center}{\large \textbf{Math 2551 Worksheet Answers: Polar Double Integrals}}
\end{center}

\begin{enumerate}
	\item $5\pi-26$
	
	\item $\dfrac{\pi}{2}(1-e^{-4}).$
	
	\item $\dfrac{3\pi}{2}-4$.
	
	\item $256\pi$
	
	\item \begin{enumerate}
		\item $I^2=\Ds \lim_{R\to\infty}\int_0^2\pi\int_0^R e^{-r^2}r\ dr\ d\theta$
		
		\item $I^2=\dfrac{\pi}{4}$, so $I=\dfrac{\sqrt{\pi}}{2}$
	\end{enumerate}
\end{enumerate}
}{}
\iftoggle{solutions}
{
Solutions go here in the same format.
}{}
% \fancyhead[C]{Sections 14.3-14.8, 15.1-15.4}
	\fancyhead[R]{\dayseventeen}

\iftoggle{questions}{
\begin{center}{\large \textbf{Math 2551 Worksheet: Exam 2 Review}}
\end{center}


\begin{enumerate}
	
	
	\item Which of the following statements are true if $f(x, y)$ is differentiable
	at $(x_0 , y_0)$? Give reasons for your answers.
	\begin{enumerate}
		\item  If $\bu$ is a unit vector, the derivative of $f$ at $(x_0 , y_0)$ in the direction
		of $\bu$ is $(f_x(x_0 , y_0)\bi + f_y(x_0 , y_0)\bj) \cdot \bu$.
		\item The derivative of $f$ at $(x_0 , y_0)$ in the direction of $\bu$ is a vector.
		\item The directional derivative of $f$ at $(x_0 , y_0)$ has its greatest value
		in the direction of $\nabla f$.
		\item At $(x_0 , y_0)$, the vector $\nabla f$ is normal to the curve $f(x, y) = f(x_0 , y_0)$.
	\end{enumerate}
	
	\item Find $dw/dt$ at $t = 0$ if $w = \sin(xy + \pi), x = e^t,$ and $y =
	\ln(t + 1).$
	
	\item Find the extreme values of $f(x, y) = x^3 + y^2$ on the circle $x^2 + y^2 = 1$.
	
	\item Test the function $f(x,y)=x^3+y^3+3x^2-3y^2$ for local maxima and minima and saddle points and find the function's value at these points.
	
	\item Find the points on the surface $xy+yz+zx-x-z^2=0$ where the tangent plane is parallel to the $xy$-plane.
	
	\item Evaluate the integral $\displaystyle \int_0^1\int_{2y}^2 4\cos(x^2)\ dx\ dy$. Describe why you made any choices you did in the course of evaluating this integral.
	
	\item If $f(x,y)\geq 2$ for all $(x,y)$, is it possible that the average value of $f(x,y)$ on a unit disk centered at the origin is $\dfrac{2}{\pi}$?
	
	\item A swimming pool is circular with a 40 foot diameter.  The depth is constant along east-west lines and increases linearly from 2 feet at the south end to 7 feet at the north end.  Find the volume of water in the pool.
	
\end{enumerate}
}{}

\iftoggle{answers}{
\begin{center}{\large \textbf{Math 2551 Worksheet Answers: Exam 2 Review}}
\end{center}

\begin{enumerate}
	
	\item All are true except b).
	
	
	\item -1
	
	\item $\pm 1$
	
	\item Saddle at $(0,0)$ with $f(0,0)=0$, local min at $(0,2)$ of $-4$, local max at $(-2,0)$ of 4, saddle at $(-2,2)$ with $f(-2,2)=0$
	
	\item $(-1/2, 1/2, 1/2)$ and $(0,1,0)$
		
	\item $\sin(4)$
	
	\item No, this is less than $f(x,y)$ at all points, so it cannot possibly be the average value.
	
	\item $1800\pi$ cubic feet
\end{enumerate}
}{}
\iftoggle{solutions}
{
Solutions go here in the same format.
}{}

% 
\fancyhead[C]{Section 15.5}
\fancyhead[R]{\dayeighteen}

\section*{\centering Chapter 15.5: Triple Integrals}
\textbf{I1: Double \& Triple Integrals.} I can set up double and triple integrals as iterated integrals over any region. I can sketch regions based on a given iterated integral.

\textbf{I2: Iterated Integrals.} I can compute iterated integrals of two and three variable functions, including applying Fubini's Theorem to change the order of integration of an iterated integral.

\subsection*{Mechanics}
\begin{enumerate}
    \item Triple integrals can compute volumes, just like double integrals can, so when might you prefer to use one over the other? Give an example of a solid whose volume is easier to compute with a double integral, and vice versa. 
    
    \item Evaluate the triple iterated integral
	   \begin{equation*}
	       \int_{-1}^1\int_{0}^4\int_0^1 (z^3-4x^2y)\ dz\ dy\ dx
	   \end{equation*}
	Recall that to evaluate the innermost integral, treat both $x$ and $y$ as constants and take the antiderivative with respect to $z$.
	
	Describe the region of integration.
       	
	\item Set up a triple iterated integral for $\iiint_E z\ dV$, where $E$ is the solid tetrahedron in the first octant bounded above by $x+y+z=1$. It may be helpful to make a sketch of the solid.
    
    \item Set up integrals that would calculate the volume of the region below, using the specified orders of integration.
	\begin{center}
		\includegraphics[scale=0.45,alt={region above the square with -1<y<0,0<x<1 under the surface z=y^2}]{15_5pic.PNG}
	\end{center}
	
	\begin{multicols}{5}
    \begin{enumerate}
		\item $dy \ dz \ dx$
		\item $dy \ dx \ dz$
		\item $dx \ dy \ dz$
		\item $dx \ dz \ dy$
		\item $dz \ dx \ dy$
	\end{enumerate}
    \end{multicols}
\end{enumerate}

\pagebreak

\subsection*{Applications}
\begin{enumerate}[resume]
    \item A solid with density $\delta(x, y, z) = 3x^2yz$ is bounded below
by the plane $z = 0$, on the sides by the elliptical cylinder
$x^2 + 4y^2 = 4$, and above by the plane $z = 2 − x$. Set up
all of the necessary triple integrals to compute its center
of mass. You do not need to compute any integrals (unless you want to). It may be helpful to sketch the solid first.

    \item The triple integral below represents the volume of a particular solid, called the \textit{Steinmetz} solid:
    \begin{equation*}
        \int_{-1}^1\int_{-\sqrt{1-x^2}}^{\sqrt{1-x^2}}\int_{-\sqrt{1-x^2}}^{\sqrt{1-x^2}}dy \ dz \ dx
    \end{equation*}
    Perform this integral, and then unravel the bounds to realize it as the intersection of two simpler solids. Though mathematician Charles Steinmetz is cited with the study of this in the 17/18th century, evidence of its study traces back to ancient Greece and China, as well as the early Renaissance. Can you spot the Steinmetz solids in the following pictures?

\begin{minipage}{.45\textwidth}
  \centering
  \includegraphics[width=.8\linewidth,alt={synagogue in rome with a Steinmetz shaped dome}]{Jewish-Synagogue-Rome.jpg}
\end{minipage}%
\begin{minipage}{.45\textwidth}
  \centering
  \includegraphics[width=.9\linewidth,alt={architectural diagram of a groined vault ceiling}]{groin vault.jpg}
\end{minipage}
\end{enumerate}
\subsection*{Extensions}
\begin{enumerate}[resume]
	
		
	\item Use a clever swap of the order of integration to evaluate 
    \begin{equation*}
    \int_0^9\int_{\sqrt{z}}^3\int_0^yz\cos(y^6)dxdydz
    \end{equation*}
    \textit{[Hint: Note that the both the function and bounds do not detect the variable $x$, so you can rearrange to take care of the $x$-integral first. Now note that $\cos(y^6)$ has no elementary antiderivative.]}
    
	
	\item Let $D$ be the region bounded by the paraboloid $z=x^2+y^2$ and the 
	plane $z=2y$, i.e. \[D=\{(x,y,z) \in \R^3 \mid x^2+y^2 \leq z \leq 2y\}.\] 
	Write triple iterated integrals in the orders $dz\ dy\ dx$ and  $dx\ dz\ dy$
	that give the volume of $D$.  Can you write a single triple iterated integral for this volume using any other orders of integration?

%%%%%%
\end{enumerate}

\iftoggle{answers}
{
	\begin{center}{\large \textbf{Math 2551 Worksheet Answers: Triple Integrals}}
	\end{center}

\begin{enumerate}
	\item $\dfrac{-58}{3}$, region of integration the rectangular prism $[-1,1]\times[0,4]\times[0,1]$ or $-1\leq x \leq 1, 0\leq y \leq 4, 0\leq z\leq 1$.

	\item Various possibilities depending on order of integration.  E.g \[\int_0^1 \int_0^{1-x} \int_0^{1-x-y}\ z\ dz\ dy\ dx \]

	\item 
	\begin{enumerate}
		\item $\displaystyle\int_0^1 \int_0^1 \int_{-1}^{-\sqrt z}\ dy \ dz \ dx $
		\item $\displaystyle\int_0^1 \int_0^1 \int_{-1}^{-\sqrt z}\ dy \ dx \ dz  $
		\item $\displaystyle\int_0^1 \int_{-1}^{-\sqrt z} \int_0^1\ dx \ dy  \ dz  $
		\item $\displaystyle\int_{-1}^0 \int_0^{y^2} \int_0^1 \ dx \ dz \ dy$
		\item $\displaystyle\int_{-1}^0  \int_0^1  \int_0^{y^2}\ dz \ dx \ dy$
	\end{enumerate}

	\item $\Ds \int_{-1}^1 \int_{1-\sqrt{1-x^2}}^{1+\sqrt{1-x^2}} \int_{x^2+y^2}^{2y} dz\ dy\ dx$.
	
		$\Ds \int_{0}^{2} \int_{y^2}^{2y} \int_{-\sqrt{z-y^2}}^{\sqrt{z-y^2}} dx\ dz\ dy$.
		
		Yes, all orders of integration result in a single iterated integral.
\end{enumerate}
}{}
\iftoggle{solutions}
{
Solutions go here in the same format.
}{}

% \fancyhead[C]{Section 15.6}
	\fancyhead[R]{\dayeighteen}
	
\iftoggle{questions}{
\begin{center}{\large \textbf{Math 2551 Worksheet: Mass and Moments}}
\end{center}

\begin{enumerate}
	
	\item Find the center of mass of a thin plate bounded by the line $y=1$ and the parabola $y=x^2$ if the density is $\delta(x,y)=y+1$.  \textit{Hint: Consider symmetry.}
	
	\begin{minipage}{0.6\textwidth}
		\item A solid of constant density $\delta(x,y,z)=2$ is bounded below by the plane $z=0$, on the sides by the elliptical cylinder $x^2+4y^2=4$, and above by the plane $z=2-x$.  Set up all of the necessary triple integrals to compute its center of mass.  You do not need to compute the center of mass.
	\end{minipage}
	\begin{minipage}{0.4\textwidth}
		\begin{center}
			\includegraphics[scale=0.5]{15_6pic.PNG}
		\end{center}
	\end{minipage}
	
	{\large \textbf{Useful Formulas}}
	
	\begin{itemize}
		\item Mass: $M=\iiint_D \delta\ dV$ or $M=\iint_R \delta\ dA$\\
		
		\item First moments (2D plate): $M_{y}=\iint_R x\delta\ dA$,\ $M_{x}=\iint_R y\delta\ dA$\\
		
		\item Center of mass (2D plate): $(\bar{x},\bar{y})=\left(\dfrac{M_{y}}{M},\dfrac{M_{x}}{M}\right)$\\
		
		\item First moments (3D solid): $M_{yz}=\iiint_D x\delta\ dV$,\ $M_{xz}=\iiint_D y\delta\ dV$,\ $M_{xy}=\iiint_D z\delta\ dV$\\
		
		\item Center of mass (3D solid): $(\bar{x},\bar{y},\bar{z})=\left(\dfrac{M_{yz}}{M},\dfrac{M_{xz}}{M},\dfrac{M_{xy}}{M}\right)$\\
	\end{itemize}
%%%%%%
\end{enumerate}
}{}

\iftoggle{answers}
{
	\begin{center}{\large \textbf{Math 2551 Worksheet Answers: Mass and Moments}}
	\end{center}

\begin{enumerate}
		
	\item $(\bar{x},\bar{y})=\left(0,\dfrac{9}{14}\right)$

	\item Mass: $\int_{-2}^2\int_{-\sqrt{1-x^2/4}}^{\sqrt{1-x^2/4}} \int_0^{2-x} 2\ dz\ dy\ dx$
	
	$M_{yz}$: $\int_{-2}^2\int_{-\sqrt{1-x^2/4}}^{\sqrt{1-x^2/4}} \int_0^{2-x} 2x\ dz\ dy\ dx$
	
	$M_{xz}$: $\int_{-2}^2\int_{-\sqrt{1-x^2/4}}^{\sqrt{1-x^2/4}} \int_0^{2-x} 2y\ dz\ dy\ dx$
	
	$M_{xy}$: $\int_{-2}^2\int_{-\sqrt{1-x^2/4}}^{\sqrt{1-x^2/4}} \int_0^{2-x} 2z\ dz\ dy\ dx$
\end{enumerate}
}{}
\iftoggle{solutions}
{
Solutions go here in the same format.
}{}

% \fancyhead[C]{Section 15.7}
	\fancyhead[R]{\daynineteen}
	
\iftoggle{questions}{
\begin{center}{\large \textbf{Math 2551 Worksheet: Triple Integrals in Cylindrical \& Spherical Coordinates}}
\end{center}

During studio, do the set up portion of these integrals only.  You can practice doing the evaluation after studio.  Both the correct triple integral and correct final answer are given in the answers for this worksheet.

\begin{enumerate}
	
	\item Convert the integral $\displaystyle \int_{-1}^1 \int_0^{\sqrt{1-y^2}} \int_0^x (x^2+y^2) \ dz \ dx \ dy$ into an integral in cylindrical coordinates, and evaluate the integral.
	
	\item Suppose $a \geq 0$. Find the volume of the region cut from the solid sphere $\rho \leq a$ by the half-planes $\theta=0$ and $\theta= \pi/6$ in the first octant.
	
	\item Find the volume of the solid that lies within the sphere $x^2+y^2+z^2=4$, above the $xy$-plane, and below the cone $z=\sqrt{x^2+y^2}$.
	
	\item Find the volume of the region bounded above by the paraboloid $z=9-x^2-y^2$, below by the $xy$-plane, and lying \emph{outside} the cylinder $x^2+y^2=1$.
	
	\item Find the volume of the solid that is between the spheres $\rho=\sqrt{2}$ and $\rho=2$, but outside of the circular cylinder $x^2+y^2=1$.  It will be helpful to draw a cross-section in a plane $\theta=c$ for this problem and to use symmetry.
	
	\item Let $D$ be the right circular cylinder whose base is the circle $r=2\sin \theta$ in the $xy$-plane and whose top lies in plane $z=4-y$. Recall that $r=2 \sin \theta$ describes a circle centered at $(0,1)$ with radius $1$ in the $xy$-plane. Using cylindrical coordinates,
	\begin{enumerate}
		\item find the volume of the region $D$.
		\item find the $\bar{x}$ component of the centroid of the region (hint: use symmetry).
	\end{enumerate}
\end{enumerate}
}{}

\iftoggle{answers}{
\begin{center}{\large \textbf{Math 2551 Worksheet Answers: Triple Integrals in Cylindrical \& Spherical Coordinates}}
\end{center}

\begin{enumerate}
	\item Integral: $\Ds \int_{-\pi/2}^{\pi/2} \int_0^1 \int_0^{r\cos(\theta)}r^3\ dz\ dr\ d\theta$
	
	Answer: $\dfrac{2}{5}.$
	
	\item Integral: $\Ds \int_0^{\pi/6} \int_0^{\pi/2} \int_0^a \rho^2\sin(\phi)\ d\rho\ d\phi\ d\theta$
	
	Answer: $\dfrac{a^3\pi}{18}.$
	
	\item Integral: $\Ds \int_0^{2\pi} \int_{\pi/4}^{\pi/2} \int_0^2 \rho^2\sin(\phi)\ d\rho\ d\phi\ d\theta$ 
	
	Answer: $\dfrac{8\sqrt{2}\pi}{3}$
	
	\item Integral: $\Ds \int_0^{2\pi} \int_1^3 \int_0^{9-r^2} r\ dz\ dr\ d\theta$
	
	Answer: $32\pi$
	
	\item Integral: $\Ds 2 \left( \int_0^{2\pi}\int_{\pi/6}^{\pi/4} \int_{\csc(\phi)}^{2} \rho^2\sin(\phi)\ d\rho\ d\phi\ d\theta +\int_0^{2\pi}\int_{\pi/4}^{\pi/2} \int_{\sqrt{2}}^{2} \rho^2\sin(\phi)\ d\rho\ d\phi\ d\theta \right)$
	
	Answer: $\dfrac{12\sqrt{3}-4}{3}\pi $.
	
	\item \begin{enumerate}
		\item Integral: $\Ds \int_0^{\pi} \int_0^{2\sin(\theta)} \int_0^{4-r\sin(\theta)} r\ dz\ dr\ d\theta$
		
		Answer: $3\pi.$
		\item $\bar{x}=0$.
	\end{enumerate}
\end{enumerate}

}{}
\iftoggle{solutions}
{
Solutions go here in the same format.
}{}

% \fancyhead[C]{Section 15.8}
	\fancyhead[R]{\daytwenty}
	
\iftoggle{questions}{
\begin{center}{\large \textbf{Math 2551 Worksheet: Change of Variables}}
\end{center}

\begin{enumerate}
	
	\item Find the Jacobian determinant of the transformation $x=e^{-r}\sin(\theta), y=e^r\cos(\theta)$.
	
	\item Find the image of the set $S$ which is the disk given by $u^2+v^2\leq 1$ under the transformation\[\bT(u,v)=\begin{bmatrix} au \\ bv\end{bmatrix}\] for some constants $a,b>0$.
	
	Give an example integration problem for which this result would be helpful.
	
	\item Find equations for a transformation $\bT$ that maps a rectangular region $S$ in the $uv$-plane whose sides are parallel to the $u$- and $v$-axes onto the region $R$ bounded by the hyperbolas $y=1/x, y=3/x$ and the lines $y=x, y=3x$ in the first quadrant.
	
	\item Solve the system \[u=2x-3y, v=-x+y \] for $x$ and $y$ in terms of $u$ and $v$.  Then find the value of the Jacobian and find the image of the parallelogram $R$ in the $xy$-plane with boundaries $x=-3, x=0, y=x,$ and $y=x+1$ under this transformation.  Sketch the transformed region in the $uv$-plane.  Use your results to rewrite the integral \[\iint_R 2(x-y)\ dx\ dy \] as an integral in $uv$-coordinates.
	
	\item Use a change of variables to compute \[ \iint_R xy\ dA, \] where $R$ is the region in the first quadrant bounded by the lines $y=x$ and $y=3x$ and the hyperbolas $xy=1, xy=3$.  
	
	Hint: Think about problem 3 above.	
	
	\item Compute $\Ds \iint_R e^{x+y}\ dA$, where $R$ is the region given by the inequality $|x|+|y|\leq 1$.
\end{enumerate}
}{}

\iftoggle{answers}
{
	\begin{center}{\large \textbf{Math 2551 Worksheet Answers: Change of Variables}}
	\end{center}

	\begin{enumerate}
		\item $\sin^2(\theta)-\cos^2(\theta)=-\cos(2\theta)$
		
		\item The elliptical region $\dfrac{x^2}{a^2}+\dfrac{y^2}{b^2}\leq 1$.\\
		
		Many possible answers: any such should involve an integration problem where the region of integration is an ellipse with major and minor axes parallel to the $x$ and $y$-axes.
		
		\item $x=\sqrt{\dfrac{u}{v}}, y=\sqrt{uv}$ maps $[1,3]\times[1,3]$ onto $R$	
		
		\item  $\displaystyle\int_0^1 \int_{-3v}^{-3v+3} -2v \ du\ dv $
		
		\item $\ln(9)$
		
		\item $e+\dfrac{1}{e}$
	\end{enumerate}
}{}
\iftoggle{solutions}
{
Solutions go here in the same format.
}{}

% \fancyhead[C]{Section 16.1}
	\fancyhead[R]{\daytwentyone}

\iftoggle{questions}{
\begin{center}{\large \textbf{Math 2551 Worksheet: Scalar Line Integrals}}
\end{center}

\begin{enumerate}
	\item For each curve, find a parameterization of the curve with the specified orientation.
	\begin{enumerate}
		\item The line segment in $\R^3$ from $(0,1,-2)$ to $(3,-1,2)$.
		
		\item The line segment in $\R^3$ from $(3,-1,2)$ to $(0,1,-2)$.
		
		\item The circle of radius 3 in $\R^2$ centered at the origin, beginning at the point $(0,-3)$ and proceeding clockwise around the circle.
		
		\item In $\R^2$, the portion of the parabola $y^2=x$ from the point $(4,2)$ to the point $(1,-1)$.\\
	\end{enumerate}
	
	For problems 2 to 4 below, do the set up of each line integral before doing any computations.
	
	\item Find the line integral of $f(x,y,z):=\sqrt{x^2+y^2}$ over the curve 
	$\br(t)= \vecf{(-4\sin t)}{+(4\cos t)}{+3t},$ $t \in [0,2\pi]$.\\
	
	\item Find the line integral of $f(x, y) =\sqrt{4x+1}$ over $C$ where $C$ is the part of the curve $x=y^2$ from the point $(4, -2)$ to $(1, 1)$.\\
	
	
	\item Let $C$ be the curve with parameterization
	\[
	\br(t)=\vecf{(e^t \cos t)}{+(e^t\sin t)}{+e^t},\quad t \in [0,\pi].
	\]
	Find the mass of $C$ if the density of a wire along $C$ is $\delta(x,y,z)=z^{-1}$.\\
	
	
	
	\textbf{Vector Field Problems:} In Chapter 13, we talked about vector-valued functions: functions whose input is a single real number and whose output is a vector in $\R^2$ or $\R^3$.  A very important related type of function in this unit is a \textbf{vector field}, which is a function that takes a point $(x,y)$ in $\R^2$ and outputs a vector in $\R^2$ or a point $(x,y,z)$ in $\R^3$ and outputs a vector in $\R^3$.
	
	\item If we want to make a decision based on what the wind is doing, then we need to keep track of not just its strength at any point, but also its direction! So a vector field is a good model here: at each point $(x,y)$, we can record the velocity vector for the wind.
	
	Suppose that given the point $(x,y)$ in the plane, we know that the wind velocity at that point is given by the vector $\bF(x,y)=\langle y, x\rangle$.  For example, at the point $(1,-1)$, the wind velocity is $\bF(1,-1)=\langle -1,1\rangle$.  Fill in the table below with the wind velocity vectors for the given points.\\
	
	\begin{tabular}{c|c}
		$(x,y)$ & $(-2,0)\quad (-1,2)\quad (0,-2)\quad (1,1)\quad (2,3)\quad (3,2)\quad (-1,0)\quad (1,3)$ \\
		\hline $\bF(x,y)$ &
	\end{tabular}
	
	\pagebreak
	
	\item Another useful way of recording this information is to plot the velocity vectors!  At each point $(x,y)$ we will draw the vector $\bF(x,y)$ starting with the tail of the vector at $(x,y)$.  This has been done for you with the point $(1,-1)$ below.  Fill in the plot with the other vectors you found in the last problem.
	
	\includegraphics[scale=0.6]{ws_21_plot.png}
	
	CalcPlot3d can graph vector fields too! Use it to check your work and see the full detail of this vector field.
	
	
	\item Let $f(x,y)=\frac{1}{2}(x-y)^2$. Find the gradient vector field, $\nabla f$, of $f$ and sketch it.
\end{enumerate}
}{}

\iftoggle{answers}{
\begin{center}{\large \textbf{Math 2551 Worksheet Answers: Scalar Line Integrals}}
\end{center}
\begin{enumerate}
	\item There are many possible correct answers!  Here are some.
	\begin{enumerate}
		\item $\br(t)=\langle 3t, -2t+1, 4t-2\rangle, \quad 0\leq t \leq 1$
		
		\item  $\br(t)=\langle -3t+3, 2t-1, -4t+2\rangle, \quad 0\leq t \leq 1$
		
		\item $\br(t)=\langle 3 \sin(t), 3\cos(t) \rangle, \quad \pi\leq t\leq 3\pi$
		
		\item $\br(t)=\langle t^2, -t\rangle \quad -2 \leq t \leq 1$
	\end{enumerate}
	\item The integral to compute is $\Ds \int_0^{2\pi} 20\ dt$ since $f(\br(t))=4$ and $\|\br'(t)\|=5$.  Its value is $40 \pi$.
	
	\item The integral to compute is $Ds \int_{-2}^1 4t^2+1\ dt$. Its value is $15$.
	
	
	\item The integral to compute is $Ds \int_0^{\pi} e^{-t}e^t\sqrt{3}\ dt$ $\sqrt{3}\pi$
	
	
	\item	
	\begin{tabular}{c|c}
		$(x,y)$ & $(-2,0)\quad (-1,2)\quad (0,-2)\quad (1,1)\quad (2,3)\quad (3,2)\quad (-1,0)\quad (1,3)$ \\
		\hline $\bF(x,y)$ & $\langle 0, -2 \rangle\quad \langle 2,-1\rangle\quad \langle -2, 0 \rangle \quad \langle 1,1 \rangle \quad \langle 3,2 \rangle \quad \langle 2,3 \rangle \quad \langle 0, -1 \rangle \quad \langle 3, 1\rangle$
	\end{tabular}
	
	\item \begin{minipage}{0.5\textwidth}Without rescaling vectors to fit better:
		
		\includegraphics[scale=0.4]{ws_21_plot_ans_noscale.png}
	\end{minipage}\begin{minipage}{0.5\textwidth}
	With rescaling:
	
	\includegraphics[scale=0.4]{ws_21_plot_ans.png}
\end{minipage}
	
	\item $\nabla f(x,y)=\langle x-y,y-x\rangle$
\end{enumerate}
}{}
\iftoggle{solutions}
{
Solutions go here in the same format.
}{}

% \fancyhead[C]{Section 16.2}
	\fancyhead[R]{\daytwentytwo}

\iftoggle{questions}{
\begin{center}{\large \textbf{Math 2551 Worksheet: Vector Line Integrals}}
\end{center}

\begin{enumerate}
	
	\item Consider the vector field $\bF$ (thin arrows) and and let $\bT$ denote the unit tangent vector to the directed curves shown below (denoted with thick arrows). Determine whether \[ \int_C \bF\cdot\bT\ ds\] is positive, negative, or zero for each directed curve $C$.  In other words, determine whether the \textit{work} done by the vector field on each curve is positive, negative, or zero.
	
	\includegraphics[scale=0.5]{16_2pic.png}
	
	
	\item Evaluate $\int_C (2x -y)\bi\cdot\ d\br$ where $C$ is parameterized by 
	$\br(t) = (t^2)\bi+(3t-2)\bj$ , $t \in [0,1]$.\\
	
	\item Find the work done by the force $\bF= xy \bi +(y-x) \bj$ over the straight line from $(1,1)$ to $(2,3)$.\\
	
	\item Consider the closed curve $C$ consisting of a semicircle and a straight line segment as follow:
	\[
	\br_1(t)=(2 \cos t)\bi+(2 \sin t)\bj, \ t \in [0,\pi], \qquad \br_2(t)=t\bi, \ t\in [-2,2]
	\]
	Let the vector field $\bF$ be given by 
	\[
	\bF(x,y) = -y^2 \bi + x^2 \bj.
	\]
	Find the circulation of $\bF$ around $C$ and the flux of $\bF$ across $C$. 
	%%
	
	\item Give an example of a non-trivial force field $\bF$ (not the zero vector at all points) and a non-trivial path $\br(t)$ (not the stationary path at a point $P$) for which the total work done moving along the path is zero.
\end{enumerate}
}{}

\iftoggle{answers}{
\begin{center}{\large \textbf{Math 2551 Worksheet Answers: Vector Line Integrals}}
\end{center}
\begin{enumerate}
	\item Top left: 0 \\
	Bottom left: Negative \\
	Center: Positive \\
	Top right: 0\\
	Bottom right: Negative
	
	
	\item 1
	
	\item $\dfrac{25}{6}$
	
	\item Circulation: $\dfrac{32}{3}$\\
	Flux: 0
	
	\item Many possible examples: the top left and top right examples from 1), $\nabla f$ for any $f$ together with a closed curve $\br$, any example where $\bF\cdot \br'(t)=0$, i.e. the field and curve are orthogonal, and more.
\end{enumerate}
}{}
\iftoggle{solutions}
{
Solutions go here in the same format.
}{}

% \fancyhead[C]{Section 16.3}
	\fancyhead[R]{\daytwentythree}
	
\iftoggle{questions}{
\begin{center}{\large \textbf{Math 2551 Worksheet: Potentials and Conservative Vector Fields}}
\end{center}

\begin{enumerate}
	\item Show that the vector field $\bF=12xy\bi+6(x^2+y^2)\bj$ is conservative using the mixed partials test, then find a potential function $f$ such that $\bF=\nabla f$.
	
	\item Find a potential function $f$ for 
	\[
	\bF(x,y,z) = \vecf{2xy}{+(x^2-z^2)}{-2yz}.
	\]
	Evaluate 
	\[
	\int_C \bF \dotp d\br
	\]
	where $C$ is any path from $(0,0,0)$ to $(1,2,3)$.
	
	\item Let $a,b,c,d,e$ be real numbers and
	\begin{align*}
		P(x,y,z) &= 3x + 7y + 2z; \\
		Q(x,y,z) &= ax+by+4z; \\
		R(x,y,z) &= cx+dy+ez.
	\end{align*}
	For which values of the constants $a,b,c,d,e$ is $\bF = \vecf{P}{+Q}{+R}$ a conservative vector field?
	
	\item Find a potential function $f$ for \[\bF(x,y,z)=\langle \dfrac{1}{y}, -\dfrac{x}{y^2}, 2z-1\rangle\] and use it to evaluate $\int_C \bF\dotp d\br$ along the curve $C: \br(t)=\langle \sqrt{t}, t+1, t^2\rangle, 0\leq t \leq 1$.
	
	\item Compute $\int_C \bF\cdot d\br$ for the vector field $\bF=\left\langle \dfrac{-y}{x^2+y^2},\dfrac{x}{x^2+y^2}\right\rangle$ where the curve C is the unit circle oriented counterclockwise.

\end{enumerate}}{}

\iftoggle{answers}
{
	\begin{center}{\large \textbf{Math 2551 Worksheet Answers: Potentials and Conservative Vector Fields}}
	\end{center}
	
	\begin{enumerate}
		\item $f(x,y)=12x^2y+2y^3$
		
		\item $f(x,y,z)= x^2y-z^2y$ and $\displaystyle \int_C\bF \cdot d\br =-16$.
		
		\item $7=a, 2=c, 4=d$, no restriction on $b$ or $e$.
		
		
		\item $f(x,y,z)= \dfrac{x}{y}+z^2-z$. \\
		$\int_C \bF\dotp d\br = 1/2$   
		
		\item $2\pi$
	\end{enumerate}
}{}
\iftoggle{solutions}
{
Solutions go here in the same format.
}{}

% \fancyhead[C]{Section 16.4}
	\fancyhead[R]{\daytwentyfour}

\iftoggle{questions}{
\begin{center}{\large \textbf{Math 2551 Worksheet: Curl, Divergence, Green's Theorem}}
\end{center}


\begin{enumerate}
	
	\item Below is a plot of a vector field $\bF(x,y)$.  Use this to decide whether the values of $\curl \bF \cdot \bk$ and $\Div \bF$ in each quadrant are positive, negative, or zero.
	
	\includegraphics[scale=0.3]{16_3-4pic.png}
	
	\item Compute $\Div \bF$ and $\curl \bF\cdot \bk$ for the vector field $\bF(x,y)=\langle \dfrac{y}{4}, 0 \rangle$, which was plotted above.
	
	%%
	\item Let $C$ be the ellipse 
	\[
	\left( \frac{x}{3} \right)^2 + \left( \frac{y}{4}\right)^2=1.
	\]
	\begin{enumerate}
		\item Parametrize this ellipse to give it a positive orientation.
		
		\item Let $\bF(x,y)=2x \bi + 2y \bj$. Use Green's theorem to find the circulation of $\bF$ around $C$ and its flux across $C$.
	\end{enumerate}  
	\item Let $R$ be the region in the $xy$-plane bounded above by the curve $y=3-x^2$ and below by the curve $y=x^4+1$. Orient this boundary positively. Let 
	\[
	\bF(x,y) = (y+e^x \ln y) \bi + (e^x/y)\bj.
	\]
	Use Green's theorem to find the circulation of $\bF$ around $C$. What happens when you try to use Green's theorem to evaluate the flux of $\bF$ across $C$? Should you use Green's theorem to evaluate the flux integral?
	
	\item Use Green’s Theorem to find the work done by the force $\bF(x,y)=\langle x(x+y),xy^2\rangle$ in moving a particle from the origin along the $x$-axis to $(1,0)$, then along the line segment to $(0,1)$, and then back to the origin along the $y$-axis.
	
	\item \textbf{Looking ahead:} Find a parameterization (a function $\br(s,t)=\langle x(s,t), y(s,t), z(s,t)\rangle$) of the plane through the origin that contains the vectors $\bi-\bj$ and $\bj-\bk$.  Linear algebra ideas may be useful.	
	
\end{enumerate}
}{}

\iftoggle{answers}
{
	\begin{center}{\large \textbf{Math 2551 Worksheet Answers: Curl, Divergence, Green's Theorem}}
	\end{center}

\begin{enumerate}
	
	\item $\nabla\cdot \bF$ is 0 in all quadrants.\\
	$(\nabla \times \bF)\cdot \bk$ is negative in all quadrants.
	
	\item $\nabla\cdot \bF = 0$\\
	$(\nabla \times \bF)\cdot \bk=-\dfrac{1}{4}$. \\
	
	
	\item\begin{enumerate}
		\item $\br(t)=\langle 3 \cos(t), 4\sin(t) \rangle$, $0\leq t \leq 2\pi$.
		
		\item Circulation: 0\\
		Flux: $48\pi$
	\end{enumerate}  
	
	\item Circulation: $-44/15$\\
	Flux: The integrand is very difficult to work with, so we should not use Green's theorem here.
	
	\item $-1/12$
	
	\item One answer (linear algebra!) $\br(s,t)=s\langle 1,-1,0\rangle+t\langle0,1,-1\rangle = \langle s,t-s,-t \rangle$, $s,t\in\R$.
\end{enumerate}
}{}
\iftoggle{solutions}
{
Solutions go here in the same format.
}{}

% \fancyhead[C]{Section 16.5}
	\fancyhead[R]{\daytwentyfive}

\iftoggle{questions}{
\begin{center}{\large \textbf{Math 2551 Worksheet: Surfaces}}
\end{center}


\begin{enumerate}
	
	\item Consider the surface cut from the parabolic cylinder $y=4-x^2$ by the planes $z=0$, $z=2$, and $y=0$. Sketch $S$ and find a parameterization of $S$.
	
	\item Consider the surface left in the hemisphere $x^2+y^2+z^2=4, \ z\geq 0$ after cutting off the lower part between the planes $z=0$ and $z=\sqrt{3}$ from the hemisphere. (In short, $S$ is the surface which is the part of $x^2+y^2+z^2=4$ above $z=\sqrt{3}$). Sketch $S$, parametrize $S$ and find the surface area of $S$. 

	\item The tangent plane at a point $P_0=(f(u_0,v_0),g(u_0,v_0),h(u_0,v_0))$ on a parameterized surface $\br(u,v)=\langle f(u,v), g(u,v) ,h(u,v)\rangle$ is the plane through $P_0$ normal to the vector $\br_u (u_0,v_0)\times\br_v(u_0,v_0)$.\\
	
	Use this to find an equation to the tangent plane of the surface parameterized by $\br(r,\theta)=(r\cos(\theta))\bi+(r\sin(\theta))\bj+r\bk, r\geq 0, 0\leq\theta\leq 2\pi$ at the point where $(r,\theta)=(2,\pi/4)$.\\
	
	What is a Cartesian equation for this surface?  Sketch it and the tangent plane.

	\item Find the area of the part of the surface $z=xy$ that lies within the cylinder $x^2+y^2=1$. \\

	\item Find the area of the surface cut from the ``nose" of the paraboloid $x=1-y^2-z^2$ by the $yz$-plane.\\
\end{enumerate}
}{}

\iftoggle{answers}{
\begin{center}{\large \textbf{Math 2551 Worksheet Answers: Surfaces}}
\end{center}

\begin{enumerate}
	
	\item One answer: $\br(u,v)=\langle u, 4-u^2, v\rangle$, $-2\leq u\leq 2, 0\leq v\leq 2$. 
	
	\item One answer: $\br(\phi,\theta)=\langle 2\sin(\phi)\cos(\theta), 2\sin(\phi)\sin(\theta),2\cos(\phi)\rangle$ with $0\leq \phi \leq \pi/6$ and $0\leq \theta \leq 2\pi$.\\
	SA: $8\pi \left(1-\frac{\sqrt{3}}{2} \right)$
	
	\item The tangent plane is $-\sqrt{2}(x-\sqrt{2})-\sqrt{2}(y-\sqrt{2})+2(z-2)=0$. This is the cone $z=\sqrt{x^2+y^2}$.
	
	\item $\dfrac{2}{3}\pi(2^{3/2}-1)$ 
	
	\item $\dfrac{\pi}{6}(5^{3/2}-1)$
\end{enumerate}

}{}
\iftoggle{solutions}
{
Solutions go here in the same format.
}{}
% \fancyhead[C]{Section 16.6}
	\fancyhead[R]{\daytwentysix}
	
\iftoggle{questions}{
\begin{center}{\large \textbf{Math 2551 Worksheet: Surface Integrals}}
\end{center}

\title{Math 2551 Worksheet: Surface Integrals}

\begin{enumerate}

\item Integrate $f(x,y,z)=yz$ over the part of the sphere $x^2+y^2+z^2=4$ that lies above the cone $z=\sqrt{x^2+y^2}$.\\

\item Find the flux of the field $\bF(x,y,z)=x^2\bi+y^2\bj+z^2\bk$ across the surface $S$ which is the boundary of the solid half-cylinder $0\leq z \leq \sqrt{1-y^2}, 0\leq x \leq 2$, with the outward orientation. \\

\item A fluid has density $870\ kg/m^3$ and flows with velocity $\bv=\langle z, y^2,x^2\rangle$, where $x,y,z$ are measured in meters and the components of $\bv$ in meters per second.  Find the rate of flow outward through the cylinder $x^2+y^2=4, 0\leq z \leq 1$.
\end{enumerate}
}{}

\iftoggle{answers}
{
	\begin{center}{\large \textbf{Math 2551 Worksheet Answers: Surface Integrals}}
	\end{center}
	
	
	\begin{enumerate}
		
		\item 0
		
		\item $\dfrac{10\pi}{3}$
		
		\item 0
		
	\end{enumerate}

}{}
\iftoggle{solutions}
{
Solutions go here in the same format.
}{}

% \fancyhead[C]{Section 16.7}
	\fancyhead[R]{\daytwentysix}
	
\iftoggle{questions}{
\begin{center}{\large \textbf{Math 2551 Worksheet: Stokes' Theorem}}
\end{center}
\title{Math 2551 Worksheet: Stokes' Theorem}

\begin{enumerate}

\item Let $H$ be the hemisphere and $P$ be the portion of a paraboloid shown below. Use Stokes' Theorem to explain why, if $\bF$ is a vector field on $\R^3$ whose components have continuous partial derivatives, we must have \[\iint_H (\nabla \times \bF) \dotp \bn\ d\sigma=\iint_P (\nabla \times \bF) \dotp \bn\ d\sigma .\]

\includegraphics[scale=0.8]{ws_27_ph.png}

\item Use Stokes' Theorem to evaluate $\iint_S (\nabla \times \bF)\dotp \bn\ d\sigma$, where $\bF=\langle x^2z^2,y^2z^2,xyz\rangle$ and $S$ is the part of the paraboloid $z=x^2+y^2$ that lies inside the cylinder $x^2+y^2=4$, oriented upward.

\item A particle moves along line segments from the origin to the points $(1,0,0), (1,2,1),(0,2,1)$, and back to the origin under the influence of the force field
\[\bF(x,y,z)=z^2\bi+2xy\bj+2y^2\bk. \]
Find the work done by the field on the particle.
\end{enumerate}
}{}

\iftoggle{answers}
{
	\begin{center}{\large \textbf{Math 2551 Worksheet Answers: Stokes' Theorem}}
	\end{center}
	
	
	\begin{enumerate}
		
		\item $H$ and $P$ have the same oriented boundary curve $C$ provided that they are oriented in the same way, so by Stokes' theorem the given integrals must be equal.
		
		\item 0
		
		\item 3
	\end{enumerate}

}{}
\iftoggle{solutions}
{
Solutions go here in the same format.
}{}

% \fancyhead[C]{Sections 15.5-15.8, 16.1-16.8}
	\fancyhead[R]{\daytwentyseven}
	
\iftoggle{questions}{
\begin{center}{\large \textbf{Math 2551 Worksheet: Review for Exam 3}}
\end{center}

\title{Math 2551 Worksheet: Review for Exam 3}

\begin{enumerate}
	
	\item Set up an iterated integral in spherical coordinates for $\displaystyle \iiint_E z^2\ dV$ where $E$ is the region between the spheres $x^2+y^2+z^2=4$ and $x^2+y^2+z^2=25$ and inside $z=-\sqrt{\frac{1}{3}(x^2+y^2)}$.
	
	
	\item Set up an integral that computes the volume of the solid which is bounded above by the cylinder $z=4-x^2$, on the sides by the cylinder $x^2+y^2=4$, and below by the $xy$-plane using 
	\begin{enumerate}
		\item Cartesian coordinates
		\item cylindrical coordinates
	\end{enumerate}
	
	Which integral would you rather evaluate and why?

	\item Find an integral that computes the mass of the wire which lies along the curve $y^2=x^3$ from $(0,0)$ to $(1,-1)$ and has density function $\rho(x,y)=2xy^2$.\\

	\item Show that the field $\bF=2x\bi-y^2\bj-\frac{4}{1+z^2}\bk$ is conservative, find a potential function, and use it to compute the integral 
	\[ \int_C 2x\ dx -y^2\ dy - \frac{4}{1+z^2}\ dz \]
	where $C$ is any path from $(0,0,0)$ to $(3,3,1)$.\\
	
	\item Compute $\int_C (6y+x)\ dx + (y+2x)\ dy$ using any method, where $C$ is the circle $(x-2)^2+(y-3)^2=4$.\\

	\item Find the flux of the field $\bF=y\bi-x\bj+\bk$ through the portion of the sphere $x^2+y^2+z^2=a^2$ in the first octant in the direction away from the origin.\\

	\item Use Stokes' theorem to show that the circulation of the field $\bF=\langle 2x, 2y, 2z\rangle$ around the boundary curve $C$ of \textbf{any} smooth orientable surface $S$ in $\R^3$ is 0. \\

	\item Find the outward flux of $\bF = (x\bi+y\bj+z\bk)/\sqrt{x^2+y^2+z^2}$ through the boundary $S$ of the ``thick sphere" $D$ given by the points satisfying $1\leq x^2+y^2+z^2\leq 4$.
\end{enumerate}
}{}

\iftoggle{answers}{
\begin{center}{\large \textbf{Math 2551 Worksheet Answers: Review for Exam 3}}
\end{center}

\begin{enumerate}
	\item $\displaystyle \int_0^{2\pi} \int_{2\pi/3}^\pi\int_2^5 \rho^4 \cos^2(\phi)\sin(\phi) d\rho\ d\phi\ d\theta$ 
	
	
	\item 
	\begin{enumerate}
		\item $\displaystyle \int_{-2}^2 \int_{-\sqrt{4-x^2}}^{\sqrt{4-x^2}} \int_0^{4-x^2}\ dz\ dy\ dx$
		\item $\displaystyle \int_{-2}^2 \int_{-\sqrt{4-x^2}}^{\sqrt{4-x^2}} \int_0^{4-x^2}\ dz\ dy\ dx$
	\end{enumerate}
	
	\item One solution: $\Ds \int_0^1 2(t)(-t^{3/2})^2\sqrt{1+(\frac{3}{2}\sqrt{t})^2}\ dt$.\\
	
	\item $f(x,y,z)=x^2-\frac{1}{3}y^3-4\arctan(z)$.
	
	Integral: $-\pi$. \\
	
	\item $-4\cdot \pi(2)^2$\\
	
	\item $\pi a^2/4$\\
	
	\item $\nabla \times \bF = \langle 0-0,-(0-0),0-0\rangle$ \\
	
	\item $12\pi$
\end{enumerate}

}{}
\iftoggle{solutions}
{
Solutions go here in the same format.
}{}
%\fancyhead[C]{Sections 12.1-6, 13.1-4, 14.1-8, 15.1-8, 16.1-8}
	\fancyhead[R]{\daytwentyeight}
	
\iftoggle{questions}{
\begin{center}{\large \textbf{Math 2551 Worksheet: Review for Final}}
\end{center}

\title{Math 2551 Worksheet: Review for Final}

\begin{enumerate}
	
	\item Find the equation of the plane through $(1,-1,3)$ parallel to the plane $3x+y+z=7$.  Is there a unique plane through $(1,-1,3)$ which is perpendicular to the plane $3x+y+z=7$.  Explain why or why not.
	
	\item Find the point on the curve \[ \br(t)=(5\sin(t))\bi+(5\cos(t))\bj+12t\bk \] at a distance $26\pi$ units along the curve from the point $(0,5,0)$ in the direction of increasing parameter $t$.
		
	\item Find the domain and range of $f(x,y)$ =$\sqrt{x^2-y}$ and identify its level curves.
	
	\item Compute $\displaystyle \lim_{(x,y)\to(0,0)} \frac{y}{x^2-y}$ or show this limit does not exist.
	
	\item Let $f(x,y,z)=xy+2yz-3xz$.  Find the tangent plane to the surface $f(x,y,z)=1$ at $(1,1,0)$ and the linearization $L(x,y,z)$ at $(1,1,0)$.
	
	\item At the point $(1,2)$, the function $f(x,y)$ has a derivative of $2$ in the direction toward $(2,2)$ and a derivative of $-2$ in the direction toward $(1,1)$.  Find $\nabla f(1,2)$ and the derivative of $f$ at $(1,2)$ in the direction toward the point $(4,6)$.	
	
	\item Find the value of the derivative of $f(x,y,z)=xy+yz+xz$ with respect to $t$ on the curve $\br(t)=\langle \cos (t), \sin (t), \cos (2t)\rangle$ at $t=1$.

	\item Find the local minima, local maxima, and saddle points of the function $f(x,y)=x^4-8x^2+3y^2-6y$.

	\item Find the extreme values of $f(x,y)=4xy-x^4-y^4+16$ on the triangular region bounded below by the line $y=-2$, above by the line $y=x$, and on the right by the line $x=2$.

	\item Find the extreme values of $f(x,y)=xy$ on the circle $x^2+y^2=1$.

	\item Sketch the region of integration and reverse the order of integration for the integral
	\[\int_0^{3/2}\int_{-\sqrt{9-4y^2}}^{\sqrt{9-4y^2}}y\ dx\ dy.\]
	
	\item Evaluate the integral \[ \int_{-1}^1\int_{-\sqrt{1-y^2}}^{\sqrt{1-y^2}} \frac{2\ dx\ dy}{(1+x^2+y^2)^2} \] by changing to polar coordinates.
	
	\pagebreak

	\item Find the centroid of the region bounded by the lines $x=2,y=2$, and the hyperbola $xy=2$ in the $xy$-plane.

	\item Find the volume of the region bounded above by the sphere $x^2+y^2+z^2=2$ and below by the paraboloid $z=x^2+y^2$.

	\item Use the transformation $u=3x+2y, v=x+4y$ to evaluate the integral \[\iint_R (3x^2+14xy+8y^2)\ dx\ dy \] where $R$ is the region in the first quadrant bounded by the lines $y=(-3/2)x+1, y=(-3/2)x+3, y=-(1/4)x,$ and $y=-(1/4)x+1)$.
	
	\item Evaluate the integral $\int_C y^2\ dx + x^2\ dy$ where $C$ is the circle $x^2+y^2=4$.
	
	\item Find the outward flux of $\bF=2xy\bi+2yz\bj+2xz\bk$ across the boundary of the cube cut from the first octant by the planes $x=1, y=1, z=1$.

	\item Find the work done by $\bF = \dfrac{x\bi+y\bj}{(x^2+y^2)^{3/2}}$ over the plane curve $\br(t)=\langle e^t \cos(t), e^t\sin(t)\rangle$ from the point $(1,0)$ to the point $(e^{2\pi},0)$.

	\item Find the flux of the field $\bF=\langle 2xy+x, xy-y\rangle$ outward across the boundary of the square bounded by $x=0,x=1,y=0,x=1$.

	\item Find the flux of $\bF = xz\bi+yz\bj+\bk$ across the upper cap cut from the sphere $x^2+y^2+z^2=25$ by the plane $z=3$, oriented away from the $xy$-plane.
\end{enumerate}
}{}

\iftoggle{answers}{
\begin{center}{\large \textbf{Math 2551 Worksheet Answers: Review for Final}}
\end{center}

\begin{enumerate}
	\item $3x+y+z=5$.  There is not a unique plane  because there is not a unique normal direction perpendicular to $\langle 3,1,1\rangle$.
	
	\item $(0,5,24\pi)$
	
	\item Domain $\{(x,y)\mid y\leq x^2\}$
	Range $[0,\infty)$
	Level curves are the parabolas $y=x^2-c^2$ for all $c\geq 0$.
	
	\item The limit does not exist
	
	\item Tangent plane: $(x-1)+(y-1)-z=0$
	Linearization: $L(x,y,z)=1+(x-1)+(y-1)-z$
	
	\item $\nabla f(1,2)=\langle 2,2 \rangle$, $Df_{\bu}(1,2)=14/5$
	
	\item $-\sin^2(1)-\sin(1)\cos(1)+\cos^2(1)+\cos(1)\cos(2)-2\cos(1)\sin(2)-2\sin(1)\sin(2)$
	
	\item $(0,1)$ saddle point, $(2,1), (-2,1)$ local minimum
	
	\item min: -32 at $(2,-2)$  max: 18 at $(1,1)$
	
	\item min: -1/2 at $(1/\pm\sqrt{2},1/\pm\sqrt{2})$ and max: 1/2 at $(1/\pm\sqrt{2},1/\mp\sqrt{2})$.
	
	\item $\Ds \int_{-3}^3 \int_0^{\sqrt{9/4-x^2/4}} y\ dy\ dx$
	
	\item $\pi$
	
	\item $(\bar{x},\bar{y})=\left(\dfrac{1}{2-\ln(4)},\dfrac{1}{2-\ln(4)}\right)$
	
	\item $V=\dfrac{\pi}{6}(8\sqrt{2}-7)$
	
	\item $\dfrac{64}{5}$
	
	\item $0$
	
	\item $3$
	
	\item $1-e^{-2\pi}$
	
	\item $\dfrac{3}{2}$
	
	\item $\dfrac{208\pi}{5}$
\end{enumerate}

}{}
\iftoggle{solutions}
{
Solutions go here in the same format.
}{}

%\fancyhead[C]{Sections 14.3-14.8, 15.1-15.4}
	\fancyhead[R]{HP Review Session}

\iftoggle{questions}{
\begin{center}{\large \textbf{Math 2551 HP Exam 2 Review}}
\end{center}


\begin{enumerate}
	
	
	\item Choose whether each statement is true or false. If the statement is 
	\textit{always} true, pick true. If the statement is \textit{ever} false, 
	pick false.  Give a reason for your answer.
		\begin{enumerate}
			\item If $f_x=f_y$ everywhere, then $f(x,y)$ is constant.
			
			\item There exists a function $f(x,y)$ with $f_x=2x+y$ and 
			$f_y=2x+2y$.
			
			\item If the temperature at a point $(x,y)$ on the floor of a room 
			is given by $T(x,y)$ and heat is being radiated out from a hot spot 
			at the origin, then if $a,b>0$ $\nabla T(a,b)$ could be $\langle 
			2,-2\rangle$.
			
			\item The integral $\iint_R y\ dA$ over the region $R: -1\leq x 
			\leq 1, 0\leq y \leq 1$ is zero.
		\end{enumerate}
	
	\item Based on the contour plot below, determine the signs of the requested 
	derivatives and draw the requested gradients.
	
	\begin{minipage}{0.45\textwidth}
		\includegraphics[scale=0.7]{hp_review_contour.png}
	\end{minipage}\begin{minipage}{0.45\textwidth}
	\begin{enumerate}
		\item $f_x(0,0)$ and $f_y(0,0)$
		\item $Df_{\bu}(2,-6)$, $u=\langle1/\sqrt{2},-1/\sqrt{2}\rangle$
		\item $\nabla f(-4,-4)$
		\item The rate of change of $f$ at $(-4,5)$ in the direction towards 
		$(4,6)$
	\end{enumerate}
	\end{minipage}
	
	\item Use the Chain Rule to find the total derivative of the composite 
	function $h=g\circ f$ at the point $(s,t)=(1,1)$, where \[ 
	f(s,t)=\begin{bmatrix}
		s^2+2t \\ 2s-t
	\end{bmatrix}\qquad g(x,y)=3x^2+4y^2.\]
	
	\item Which of the following gaurantees a saddle point of a continuously 
	differentiable function $f(x,y)$ at $(a,b)$?
	\begin{enumerate}
		\item $f_{xx}$ and $f_{yy}$ have the same sign at $(a,b)$
		\item $f_{xx}$ and $f_{yy}$ have opposite signs at $(a,b)$
		\item $f_{xy}$ is negative at $(a,b)$
		\item None of the above.
	\end{enumerate}
	
	\item Find and classify the critical points of $xy-2x-2y-x^2-y^2$. 
	
	\item Find the extreme values of $f(x,y,z)=2x+2y+z$ subject to the 
	constraint $x^2+y^2+z^2=9$. Interpret your results geometrically.
	
	\item Sketch the region of integration, set up iterated integrals for both 
	orders of integration, then evaluate using the easier order and explain why 
	it is easier. \[\iint_R y^2 e^{xy}\ dA, \quad R: x\geq 0, x\leq y \leq 4.\]
\end{enumerate}
}{}

\iftoggle{answers}{
\begin{center}{\large \textbf{Math 2551 HP Exam 2 Review Answers}}
\end{center}

\begin{enumerate}
	
	\item All are true except b).
	
	
	\item -1
	
	\item $\pm 1$
	
	\item Saddle at $(0,0)$ with $f(0,0)=0$, local min at $(0,2)$ of $-4$, local max at $(-2,0)$ of 4, saddle at $(-2,2)$ with $f(-2,2)=0$
	
	\item $(-1/2, 1/2, 1/2)$ and $(0,1,0)$
		
	\item $\sin(4)$
	
	\item No, this is less than $f(x,y)$ at all points, so it cannot possibly be the average value.
	
	\item $1800\pi$ cubic feet
\end{enumerate}
}{}
\iftoggle{solutions}
{
Solutions go here in the same format.
}{}

%\fancyhead[C]{Sections 15.5-8, 16.1-8}
	\fancyhead[R]{HP Review Session}

\iftoggle{questions}{
\begin{center}{\large \textbf{Math 2551 HP Exam 3 Review}}
\end{center}


\begin{enumerate}
	\item Set up integrals for the volume of the solid $S$ bounded by the three coordinate planes, bounded above by the plane $x+y+z=2$, and bounded below by the plane $z=x+y$ using two different orders of integration.
	
	\item Set up integrals that compute the volumes of the solids:
	\begin{enumerate}
		\item Bounded by $z=x^2+y^2$ and $z^2=4(x^2+y^2)$
		\item Bounded above by the hemisphere $x^2+y^2+z^2=1, z\geq 0$ and below by the \textit{cardioid of revolution} $\rho=1+\cos(\varphi)$
	\end{enumerate}
	
	\item Use change of variables with $u=(x+y)/2, v=(x-y)/2$ to compute \[\iint_R \sin(\frac{x+y}{2})\cos(\frac{x-y}{2})\ dA,\] where $R$ is the triangle with vertices $(0,0), (2,0),$ and $(1,1)$.
	
	\item Use an appropriate method to compute each of the following:
	\begin{enumerate}
			
		\item The flux of the field $\bF(x,y,z)=\langle x,2y,3z\rangle$ through the portion of the plane $x+y/2+z/3=0$ in the first octant oriented away from the origin.
		\item The work done by $\bF(x,y)=x\bi-y\bj$ on a particle moving from $(-1,2)$ to $(1,2)$ along $y=x^2+1$.
		\item The circulation of the field $\bF(x,y)=y\bi-x\bj$ around a circle of radius $4$ about the origin, oriented clockwise.
		\item The flux of the field $\bF(x,y)=x\bi-y\bj$ across the portion of the curve $y=x^2+1$ from $(-1,2)$ to $(1,2)$ with normal vector upward.
		\item The work done by $\bF(x,y)=xy^2\bi+yx^2\bj$ on a particle moving from $(-1,2)$ to $(1,2)$ along $y=x^2+1$.
		\item The circulation of the field $\bF(x,y)=\langle -2z, 3x, -y\rangle$ around the boundary of surface which is the portion of the plane $x+y/2+z/3=0$ in the first octant oriented from $(1,0,0)$ to $(0,2,0)$ to $(0,0,3)$.
		\item The flux of the field $\bF(x,y,z)=\langle x^3,y^3,z^3\rangle$ out of the sphere $x^2+y^2+z^2=1$.
		\item The surface area of the part of the cylinder $x^2+y^2=$ below the plane $x+2y+z=6$ and above the $xy$-plane (there are at least two methods for this).
	\end{enumerate}
		
\end{enumerate}
}{}

\iftoggle{answers}{
\begin{center}{\large \textbf{Math 2551 HP Exam 3 Review Answers}}
\end{center}

\begin{enumerate}
	
	\item 
\end{enumerate}
}{}
\iftoggle{solutions}
{
Solutions go here in the same format.
}{}



\end{document}

\item \textbf{G1: Lines and Planes.} I can describe lines using the vector equation of a line. I can describe planes using the general equation of a plane. I can find the equations of planes using a point and a normal vector. I can find the intersections of lines and planes.  I can describe the relationships of lines and planes to each other. I can solve problems with lines and planes.
		
		\item \textbf{G2: Calculus of Curves.} I can compute tangent vectors to parametric curves and their velocity, speed, and acceleration. I can find equations of tangent lines to parametric curves. I can solve initial value problems for motion on parametric curves.
		
		\item \textbf{G3: Geometry of Curves.} I can compute the arc length of a curve in two or three dimensions and apply arc length to solve problems. I can compute normal vectors and curvature for curves in two and three dimensions.  I can interpret these objects geometrically and in applications.
		
		\item \textbf{G4: Surfaces.} I can identify standard quadric surfaces including: spheres, ellipsoids, elliptic paraboloids, hyperboloids, cones, and hyperbolic paraboloids. I can match graphs of functions of two variables to their equations and contour plots and determine their domains and ranges.

		\item \textbf{G5: Parameterization.} I can find parametric equations for common curves, such as line segments, graphs of functions of one variable, circles, and ellipses.  I can match given parametric equations to Cartesian equations and graphs. I can parameterize common surfaces, such as planes, quadric surfaces, and functions of two variables.

        \item \textbf{D1: Limits of Functions.} I can calculate the limits of some functions of two variables or and apply the Two-Path Test to determine if they do not exist. I can state the definition of continuity for functions of multiple variables.
        
		\item \textbf{D2: Computing Derivatives.} I can compute partial derivatives, total derivatives, directional derivatives, and gradients. I can use the Chain Rule for multivariable functions to compute derivatives of composite functions.
		
		\item \textbf{D3: Tangent Planes and Linear Approximations.} I can find equations for tangent planes to surfaces and linear approximations of functions at a given point and apply these to solve problems.
		
		\item \textbf{D4: Optimization.} I can locate and classify critical points of functions of two variables. I can find absolute maxima and minima on closed bounded sets. I can use the method of Lagrange multipliers to maximize and minimize functions of two or three variables subject to constraints. I can interpret the results of my calculations to solve problems.
		
		\item \textbf{I1: Double \& Triple Integrals.} I can set up double and triple integrals as iterated integrals over any region. I can sketch regions based on a given iterated integral.  
		
		\item \textbf{I2: Iterated Integrals.} I can compute iterated integrals of two and three variable functions, including applying Fubini's Theorem to change the order of integration of an iterated integral.
		
		\item \textbf{I3: Change of Variables.} I can use polar, cylindrical, and spherical coordinates to transform double and triple integrals and can sketch regions based on given polar, cylindrical, and spherical iterated integrals. I can use general change of variables to transform double and triple integrals for easier calculation.  I can choose the most appropriate coordinate system to evaluate a specific integral.
		
		\item \textbf{A1: Interpreting Derivatives.} I can interpret the meaning of a partial derivative, a gradient, or a directional derivative of a function at a given point in a specified direction, including in the context of a graph or a contour plot.
		
		\item \textbf{A2: Integral Applications.}  I can use multiple integrals to solve physical problems, such as finding area, average value, volume, or the mass or center of mass of a lamina or solid. I can interpret mass, center of mass, work, flow, circulation, flux, and surface area in terms of line and/or surface integrals, as appropriate.
		
		\item \textbf{V1: Line Integrals.} I can set up and evaluate scalar and vector field line integrals in two and three dimensions.
					
		\item \textbf{V2: Conservative Vector Fields.} I can test for conservative vector fields and find potential functions. I can state and apply the Fundamental Theorem of Line Integrals.
					
		\item \textbf{V3: Generalizations of the FTC.} I can state and apply Green's Theorem, Stokes' Theorem and the Divergence Theorem to solve problems in two and three dimensions. I can choose which theorem is appropriate for different integrals. I can compute curl and divergence of vector fields.
					
		\item \textbf{V4: Surface Integrals.} I can set up and compute surface integrals for scalar and vector valued functions.