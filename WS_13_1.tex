
\fancyhead[C]{Section 13.1}
\fancyhead[R]{\dayfour}

\iftoggle{questions}{
\begin{center}{\large \textbf{Math 2551 Worksheet: Curves in Space and Their Tangents}}
\end{center}


\begin{enumerate}

	\item Describe the graph of the curve $\br(t)=\langle t\cos(t), t\sin(t), t\rangle$, $t\in\R$.\\
	
	\item Find a vector-valued function for the curve of intersection of the cylinder $x^2+y^2=9$ and the plane $y+z=2$.
	
	\textit{Hint: How could you parameterize the circle $x^2+y^2=9$ in the plane?}\\
	
	\item What is the difference between the parameteric curves $\bff(t)=\langle t, t,t^2\rangle, \bg(t)=\langle t^2, t^2, t^4\rangle$, and $\bh(t)=\langle \sin(t),\sin(t),\sin^2(t)\rangle$ as $t$ runs over all real numbers?\\
	
	\item With a parametric plot and a set of $t$ values, we can associate a `direction'.  For example, the curve $\langle \cos(t),\sin(t)\rangle$, $t\in[0,2\pi]$ is the unit circle traced counterclockwise.  How can we change a set of given parametric equations and $t$ values to get the same curve, only traced backwards?
	
	\item The motion of a particle in the $xy$-plane at time $t$ is described by the vector function
	\[\br(t)= e^{t}\bi + \frac{2}{9}e^{2t}\bj\]
	\begin{enumerate}
		\item Find an equation in $x$ and $y$ whose graph is the path of the particle. Consider how $y(t)$ is related to $x(t)$ and what values $x(t)$ takes on.
		
		\item Find the particle's velocity and acceleration vectors at $t=\ln(3).$
		
		\item Sketch the path of the particle and include the particle's velocity and acceleration vectors at $t=\ln(3).$
	\end{enumerate}
	
	\item Find the parametric equations for the line that is tangent to the curve \\
	\[\vect r(t)=\left\langle \ln t, \frac{t-1}{t+2}, t \ln t \right\rangle, \text{ at } t = 1.\]
	
	\item Determine the point at which $\bff(t)=\langle t, t^2,t^3\rangle$ and $\bg(t)=\langle \cos(t), \cos(2t), t+1\rangle$ intersect, and find the angle between the curves at that point. (Hint: You'll need to set this up like the line intersection problems you've seen before, writing one in $s$ and one in $t$). 
	
	If these two functions were the trajectories of two bumblebees on the same scale of time, would the bees collide at their point of intersection? Explain.
	
	\item Find the equation of the plane perpendicular to the curve $\langle \cos(t),\sin(t),\cos(6t)\rangle$ when $t=\pi/4$.
\end{enumerate}
}{}

\iftoggle{answers}{
\fancyhead[R]{\dayfour}
\begin{center}{\large \textbf{Math 2551 Worksheet Answers: Curves in Space and Their Tangents}}
\end{center}


\begin{enumerate}

	\item This curve's graph is a spiral, narrowing to a point at the origin when $t=0$ and widening outward around the $z$-axis for larger/smaller $t$.\\
	
	\item $\br(t)=\langle 3\cos(t), 3\sin(t),2-3\sin(t)\rangle, 0\leq t\leq 2\pi$.\\
	
	\item  All three functions describe part of the same set of points in $\R^3$, which lie above the line $y=x$ in the $xy$-plane and form a parabola in the plane $x=y$.  $\bff$ traces out all of the points on this parabola, $\bg$ only those in the first octant, and $\bh$ only those which lie above the square $[-1,1]\times[-1,1]$.\\
	
	\item Many possible answers; depending on the domain and functions involved.  If the domain is bounded, e.g. $[a,b]$, then letting $s=b+(a-b)t$ and taking $\br(s)$ as the new parametric equations works. If the domain is $(-\infty,\infty)$, we can just let $s=-t$.
	
	
	\item 
	\begin{enumerate}
		\item $y=\dfrac{2}{9}x^2$ for $x>0$
		
		\item $\bv(\ln(3))=3\bi+4\bj$
		
		$\ba(\ln(3))=3\bi+8\bj$.
	\end{enumerate}
	
	\item $x(s)=s, y(s)=\dfrac{s}{3}, z(s)=s$
	
	
	
	\item $(1,1,1)$ (where the first parameter is 1 and the second is 0).  The angle is $\arccos(3/\sqrt{14})$.  The bees would not collide, since the first bee reaches the point at $t=1$ and the second bee at $t=0$.
	
	\item $-\dfrac{1}{\sqrt{2}}(x-\dfrac{1}{\sqrt{2}})+\dfrac{1}{\sqrt{2}}(y-\dfrac{1}{\sqrt{2}})+6(z-0)=0$
	
	OR $x-y-6\sqrt{2}z=0$
\end{enumerate}
}{}
\iftoggle{solutions}
{
Solutions go here in the same format.
}{}
