
\fancyhead[C]{Section 15.3}
\fancyhead[R]{\dayfifteen}

\section*{\centering Chapter 15.3: More Double Integrals}
\textbf{I1: Double \& Triple Integrals.} I can set up double and triple integrals as iterated integrals over any region. I can sketch regions based on a given iterated integral.

\textbf{I2: Iterated Integrals.} I can compute iterated integrals of two and three variable functions, including applying Fubini's Theorem to change the order of integration of an iterated integral.

\subsection*{Mechanics}
\begin{enumerate}
    	\item \question{
         Consider the function $f(x,y)=xy$. Without performing any computations, do you think the average value of $f$ is larger over the square $0\leq x \leq 1, 0\leq y\leq 1$, or over the quarter circle $x^2+y^2\leq 1$ \textit{in the first quadrant}? Verify your guess by integrating
        }
        {% answer here
        
        }
        {% solution here
        
        }
       
        \item \question{
        A metal triangular plate with vertices $(0,0)$, $(2,0)$ and $(2,4)$ has temperature equal to $C(x,y) = xe^{xy}$ degrees Celsius. Compute the average temperature of the plate. \textit{[Hint: Choose a favourable order of integration.]}
        }
        {% answer here
        
        }
        {% solution here
        
        }
\end{enumerate}
\subsection*{Applications}
\begin{enumerate}[resume]
    \item \question{
        If $f(x,y)=100(y+1)$ represents the population density in people per square mile of a planar region on Earth, where $x$ and $y$ are measured in miles, find the number of people in the region bounded by the curves $x=y^2$ and $x=2y-y^2$.
        }
        {% answer here
        
        }
        {% solution here
        
        }
    \item \question{
        A rectangular can of Pringles chips may be modelled by the prism $0\leq x \leq 1$, $0\leq y \leq 1$ and $0\leq z\leq 5$. Assuming that the Pringles container is filled up with chips until the surface $z=x^2-y^2+3$, are there more chips or air in the can? \textit{[Note: The Pringles enthusiast may complain that their containers are supposed to be cylinders, not prisms. This nuance will be addressed when we work with polar coordinates.]}
        }
        {% answer here
        
        }
        {% solution here
        
        }
\end{enumerate}
\subsection*{Extensions}
\begin{enumerate}[resume]
	\item \question{
        An organism can be initially described as the solid with base $[0,1]\times [0,1]$ and height $z = e^{x+y}$. Suppose that the base of this organism grows at a rate of $t$ units per second in both the positive $x$ and positive $y$ directions. Compute the rate of change of the volume of the organism at $t=4$ seconds. \textit{[Hint: Set up an integral expression for the volume in terms of $t$, evaluate the integral, then differentiate with respect to $t$.]}
        }
        {% answer here
        
        }
        {% solution here
        
        }
	
\end{enumerate}
{}

\iftoggle{answers}{
\begin{center}{\large \textbf{Math 2551 Worksheet 16 Answers: Applications, Polar Double Integrals}}
\end{center}

\begin{enumerate}
	\item Answers will vary a bit through the estimation process
	
	4 subdivisions: $31.75 \leq T_{avg} \leq 52.5$\\
	16 subdivisions: $33.18  \leq T_{avg} \leq 50.06$\\
	25 subdivisions: $36.32  \leq T_{avg} \leq 49.8$\\
	
	Colorado is a rectangle, which makes it easy to subdivide. Wyoming would also work well.
	
	
	\item On square: $f_{avg}=\frac{1}{1}\cdot \frac{1}{4}=\frac{1}{4}$\\
	
	On quarter circle: $f_{avg}=\frac{1}{\pi/4}\cdot \frac{1}{8}=\frac{1}{2\pi}$ 
	\item 50 people
\end{enumerate}
}{}
\iftoggle{solutions}
{
Solutions go here in the same format.
}{}