\fancyhead[C]{Sections 15.5-8, 16.1-8}
	\fancyhead[R]{HP Review Session}

\iftoggle{questions}{
\begin{center}{\large \textbf{Math 2551 HP Exam 3 Review}}
\end{center}


\begin{enumerate}
	\item Set up integrals for the volume of the solid $S$ bounded by the three coordinate planes, bounded above by the plane $x+y+z=2$, and bounded below by the plane $z=x+y$ using two different orders of integration.
	
	\item Set up integrals that compute the volumes of the solids:
	\begin{enumerate}
		\item Bounded by $z=x^2+y^2$ and $z^2=4(x^2+y^2)$
		\item Bounded above by the hemisphere $x^2+y^2+z^2=1, z\geq 0$ and below by the \textit{cardioid of revolution} $\rho=1+\cos(\varphi)$
	\end{enumerate}
	
	\item Use change of variables with $u=(x+y)/2, v=(x-y)/2$ to compute \[\iint_R \sin(\frac{x+y}{2})\cos(\frac{x-y}{2})\ dA,\] where $R$ is the triangle with vertices $(0,0), (2,0),$ and $(1,1)$.
	
	\item Use an appropriate method to compute each of the following:
	\begin{enumerate}
			
		\item The flux of the field $\bF(x,y,z)=\langle x,2y,3z\rangle$ through the portion of the plane $x+y/2+z/3=0$ in the first octant oriented away from the origin.
		\item The work done by $\bF(x,y)=x\bi-y\bj$ on a particle moving from $(-1,2)$ to $(1,2)$ along $y=x^2+1$.
		\item The circulation of the field $\bF(x,y)=y\bi-x\bj$ around a circle of radius $4$ about the origin, oriented clockwise.
		\item The flux of the field $\bF(x,y)=x\bi-y\bj$ across the portion of the curve $y=x^2+1$ from $(-1,2)$ to $(1,2)$ with normal vector upward.
		\item The work done by $\bF(x,y)=xy^2\bi+yx^2\bj$ on a particle moving from $(-1,2)$ to $(1,2)$ along $y=x^2+1$.
		\item The circulation of the field $\bF(x,y)=\langle -2z, 3x, -y\rangle$ around the boundary of surface which is the portion of the plane $x+y/2+z/3=0$ in the first octant oriented from $(1,0,0)$ to $(0,2,0)$ to $(0,0,3)$.
		\item The flux of the field $\bF(x,y,z)=\langle x^3,y^3,z^3\rangle$ out of the sphere $x^2+y^2+z^2=1$.
		\item The surface area of the part of the cylinder $x^2+y^2=$ below the plane $x+2y+z=6$ and above the $xy$-plane (there are at least two methods for this).
	\end{enumerate}
		
\end{enumerate}
}{}

\iftoggle{answers}{
\begin{center}{\large \textbf{Math 2551 HP Exam 3 Review Answers}}
\end{center}

\begin{enumerate}
	
	\item 
\end{enumerate}
}{}
\iftoggle{solutions}
{
Solutions go here in the same format.
}{}
